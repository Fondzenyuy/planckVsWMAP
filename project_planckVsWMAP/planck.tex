\documentclass[11pt]{article}

    \usepackage[breakable]{tcolorbox}
    \usepackage{parskip} % Stop auto-indenting (to mimic markdown behaviour)
    
    \usepackage{iftex}
    \ifPDFTeX
    	\usepackage[T1]{fontenc}
    	\usepackage{mathpazo}
    \else
    	\usepackage{fontspec}
    \fi

    % Basic figure setup, for now with no caption control since it's done
    % automatically by Pandoc (which extracts ![](path) syntax from Markdown).
    \usepackage{graphicx}
    % Maintain compatibility with old templates. Remove in nbconvert 6.0
    \let\Oldincludegraphics\includegraphics
    % Ensure that by default, figures have no caption (until we provide a
    % proper Figure object with a Caption API and a way to capture that
    % in the conversion process - todo).
    \usepackage{caption}
    \DeclareCaptionFormat{nocaption}{}
    \captionsetup{format=nocaption,aboveskip=0pt,belowskip=0pt}

    \usepackage{float}
    \floatplacement{figure}{H} % forces figures to be placed at the correct location
    \usepackage{xcolor} % Allow colors to be defined
    \usepackage{enumerate} % Needed for markdown enumerations to work
    \usepackage{geometry} % Used to adjust the document margins
    \usepackage{amsmath} % Equations
    \usepackage{amssymb} % Equations
    \usepackage{textcomp} % defines textquotesingle
    % Hack from http://tex.stackexchange.com/a/47451/13684:
    \AtBeginDocument{%
        \def\PYZsq{\textquotesingle}% Upright quotes in Pygmentized code
    }
    \usepackage{upquote} % Upright quotes for verbatim code
    \usepackage{eurosym} % defines \euro
    \usepackage[mathletters]{ucs} % Extended unicode (utf-8) support
    \usepackage{fancyvrb} % verbatim replacement that allows latex
    \usepackage{grffile} % extends the file name processing of package graphics 
                         % to support a larger range
    \makeatletter % fix for old versions of grffile with XeLaTeX
    \@ifpackagelater{grffile}{2019/11/01}
    {
      % Do nothing on new versions
    }
    {
      \def\Gread@@xetex#1{%
        \IfFileExists{"\Gin@base".bb}%
        {\Gread@eps{\Gin@base.bb}}%
        {\Gread@@xetex@aux#1}%
      }
    }
    \makeatother
    \usepackage[Export]{adjustbox} % Used to constrain images to a maximum size
    \adjustboxset{max size={0.9\linewidth}{0.9\paperheight}}

    % The hyperref package gives us a pdf with properly built
    % internal navigation ('pdf bookmarks' for the table of contents,
    % internal cross-reference links, web links for URLs, etc.)
    \usepackage{hyperref}
    % The default LaTeX title has an obnoxious amount of whitespace. By default,
    % titling removes some of it. It also provides customization options.
    \usepackage{titling}
    \usepackage{longtable} % longtable support required by pandoc >1.10
    \usepackage{booktabs}  % table support for pandoc > 1.12.2
    \usepackage[inline]{enumitem} % IRkernel/repr support (it uses the enumerate* environment)
    \usepackage[normalem]{ulem} % ulem is needed to support strikethroughs (\sout)
                                % normalem makes italics be italics, not underlines
    \usepackage{mathrsfs}
    

    
    % Colors for the hyperref package
    \definecolor{urlcolor}{rgb}{0,.145,.698}
    \definecolor{linkcolor}{rgb}{.71,0.21,0.01}
    \definecolor{citecolor}{rgb}{.12,.54,.11}

    % ANSI colors
    \definecolor{ansi-black}{HTML}{3E424D}
    \definecolor{ansi-black-intense}{HTML}{282C36}
    \definecolor{ansi-red}{HTML}{E75C58}
    \definecolor{ansi-red-intense}{HTML}{B22B31}
    \definecolor{ansi-green}{HTML}{00A250}
    \definecolor{ansi-green-intense}{HTML}{007427}
    \definecolor{ansi-yellow}{HTML}{DDB62B}
    \definecolor{ansi-yellow-intense}{HTML}{B27D12}
    \definecolor{ansi-blue}{HTML}{208FFB}
    \definecolor{ansi-blue-intense}{HTML}{0065CA}
    \definecolor{ansi-magenta}{HTML}{D160C4}
    \definecolor{ansi-magenta-intense}{HTML}{A03196}
    \definecolor{ansi-cyan}{HTML}{60C6C8}
    \definecolor{ansi-cyan-intense}{HTML}{258F8F}
    \definecolor{ansi-white}{HTML}{C5C1B4}
    \definecolor{ansi-white-intense}{HTML}{A1A6B2}
    \definecolor{ansi-default-inverse-fg}{HTML}{FFFFFF}
    \definecolor{ansi-default-inverse-bg}{HTML}{000000}

    % common color for the border for error outputs.
    \definecolor{outerrorbackground}{HTML}{FFDFDF}

    % commands and environments needed by pandoc snippets
    % extracted from the output of `pandoc -s`
    \providecommand{\tightlist}{%
      \setlength{\itemsep}{0pt}\setlength{\parskip}{0pt}}
    \DefineVerbatimEnvironment{Highlighting}{Verbatim}{commandchars=\\\{\}}
    % Add ',fontsize=\small' for more characters per line
    \newenvironment{Shaded}{}{}
    \newcommand{\KeywordTok}[1]{\textcolor[rgb]{0.00,0.44,0.13}{\textbf{{#1}}}}
    \newcommand{\DataTypeTok}[1]{\textcolor[rgb]{0.56,0.13,0.00}{{#1}}}
    \newcommand{\DecValTok}[1]{\textcolor[rgb]{0.25,0.63,0.44}{{#1}}}
    \newcommand{\BaseNTok}[1]{\textcolor[rgb]{0.25,0.63,0.44}{{#1}}}
    \newcommand{\FloatTok}[1]{\textcolor[rgb]{0.25,0.63,0.44}{{#1}}}
    \newcommand{\CharTok}[1]{\textcolor[rgb]{0.25,0.44,0.63}{{#1}}}
    \newcommand{\StringTok}[1]{\textcolor[rgb]{0.25,0.44,0.63}{{#1}}}
    \newcommand{\CommentTok}[1]{\textcolor[rgb]{0.38,0.63,0.69}{\textit{{#1}}}}
    \newcommand{\OtherTok}[1]{\textcolor[rgb]{0.00,0.44,0.13}{{#1}}}
    \newcommand{\AlertTok}[1]{\textcolor[rgb]{1.00,0.00,0.00}{\textbf{{#1}}}}
    \newcommand{\FunctionTok}[1]{\textcolor[rgb]{0.02,0.16,0.49}{{#1}}}
    \newcommand{\RegionMarkerTok}[1]{{#1}}
    \newcommand{\ErrorTok}[1]{\textcolor[rgb]{1.00,0.00,0.00}{\textbf{{#1}}}}
    \newcommand{\NormalTok}[1]{{#1}}
    
    % Additional commands for more recent versions of Pandoc
    \newcommand{\ConstantTok}[1]{\textcolor[rgb]{0.53,0.00,0.00}{{#1}}}
    \newcommand{\SpecialCharTok}[1]{\textcolor[rgb]{0.25,0.44,0.63}{{#1}}}
    \newcommand{\VerbatimStringTok}[1]{\textcolor[rgb]{0.25,0.44,0.63}{{#1}}}
    \newcommand{\SpecialStringTok}[1]{\textcolor[rgb]{0.73,0.40,0.53}{{#1}}}
    \newcommand{\ImportTok}[1]{{#1}}
    \newcommand{\DocumentationTok}[1]{\textcolor[rgb]{0.73,0.13,0.13}{\textit{{#1}}}}
    \newcommand{\AnnotationTok}[1]{\textcolor[rgb]{0.38,0.63,0.69}{\textbf{\textit{{#1}}}}}
    \newcommand{\CommentVarTok}[1]{\textcolor[rgb]{0.38,0.63,0.69}{\textbf{\textit{{#1}}}}}
    \newcommand{\VariableTok}[1]{\textcolor[rgb]{0.10,0.09,0.49}{{#1}}}
    \newcommand{\ControlFlowTok}[1]{\textcolor[rgb]{0.00,0.44,0.13}{\textbf{{#1}}}}
    \newcommand{\OperatorTok}[1]{\textcolor[rgb]{0.40,0.40,0.40}{{#1}}}
    \newcommand{\BuiltInTok}[1]{{#1}}
    \newcommand{\ExtensionTok}[1]{{#1}}
    \newcommand{\PreprocessorTok}[1]{\textcolor[rgb]{0.74,0.48,0.00}{{#1}}}
    \newcommand{\AttributeTok}[1]{\textcolor[rgb]{0.49,0.56,0.16}{{#1}}}
    \newcommand{\InformationTok}[1]{\textcolor[rgb]{0.38,0.63,0.69}{\textbf{\textit{{#1}}}}}
    \newcommand{\WarningTok}[1]{\textcolor[rgb]{0.38,0.63,0.69}{\textbf{\textit{{#1}}}}}
    
    
    % Define a nice break command that doesn't care if a line doesn't already
    % exist.
    \def\br{\hspace*{\fill} \\* }
    % Math Jax compatibility definitions
    \def\gt{>}
    \def\lt{<}
    \let\Oldtex\TeX
    \let\Oldlatex\LaTeX
    \renewcommand{\TeX}{\textrm{\Oldtex}}
    \renewcommand{\LaTeX}{\textrm{\Oldlatex}}
    % Document parameters
    % Document title
    \title{planck}
    
    
    
    
    
% Pygments definitions
\makeatletter
\def\PY@reset{\let\PY@it=\relax \let\PY@bf=\relax%
    \let\PY@ul=\relax \let\PY@tc=\relax%
    \let\PY@bc=\relax \let\PY@ff=\relax}
\def\PY@tok#1{\csname PY@tok@#1\endcsname}
\def\PY@toks#1+{\ifx\relax#1\empty\else%
    \PY@tok{#1}\expandafter\PY@toks\fi}
\def\PY@do#1{\PY@bc{\PY@tc{\PY@ul{%
    \PY@it{\PY@bf{\PY@ff{#1}}}}}}}
\def\PY#1#2{\PY@reset\PY@toks#1+\relax+\PY@do{#2}}

\@namedef{PY@tok@w}{\def\PY@tc##1{\textcolor[rgb]{0.73,0.73,0.73}{##1}}}
\@namedef{PY@tok@c}{\let\PY@it=\textit\def\PY@tc##1{\textcolor[rgb]{0.25,0.50,0.50}{##1}}}
\@namedef{PY@tok@cp}{\def\PY@tc##1{\textcolor[rgb]{0.74,0.48,0.00}{##1}}}
\@namedef{PY@tok@k}{\let\PY@bf=\textbf\def\PY@tc##1{\textcolor[rgb]{0.00,0.50,0.00}{##1}}}
\@namedef{PY@tok@kp}{\def\PY@tc##1{\textcolor[rgb]{0.00,0.50,0.00}{##1}}}
\@namedef{PY@tok@kt}{\def\PY@tc##1{\textcolor[rgb]{0.69,0.00,0.25}{##1}}}
\@namedef{PY@tok@o}{\def\PY@tc##1{\textcolor[rgb]{0.40,0.40,0.40}{##1}}}
\@namedef{PY@tok@ow}{\let\PY@bf=\textbf\def\PY@tc##1{\textcolor[rgb]{0.67,0.13,1.00}{##1}}}
\@namedef{PY@tok@nb}{\def\PY@tc##1{\textcolor[rgb]{0.00,0.50,0.00}{##1}}}
\@namedef{PY@tok@nf}{\def\PY@tc##1{\textcolor[rgb]{0.00,0.00,1.00}{##1}}}
\@namedef{PY@tok@nc}{\let\PY@bf=\textbf\def\PY@tc##1{\textcolor[rgb]{0.00,0.00,1.00}{##1}}}
\@namedef{PY@tok@nn}{\let\PY@bf=\textbf\def\PY@tc##1{\textcolor[rgb]{0.00,0.00,1.00}{##1}}}
\@namedef{PY@tok@ne}{\let\PY@bf=\textbf\def\PY@tc##1{\textcolor[rgb]{0.82,0.25,0.23}{##1}}}
\@namedef{PY@tok@nv}{\def\PY@tc##1{\textcolor[rgb]{0.10,0.09,0.49}{##1}}}
\@namedef{PY@tok@no}{\def\PY@tc##1{\textcolor[rgb]{0.53,0.00,0.00}{##1}}}
\@namedef{PY@tok@nl}{\def\PY@tc##1{\textcolor[rgb]{0.63,0.63,0.00}{##1}}}
\@namedef{PY@tok@ni}{\let\PY@bf=\textbf\def\PY@tc##1{\textcolor[rgb]{0.60,0.60,0.60}{##1}}}
\@namedef{PY@tok@na}{\def\PY@tc##1{\textcolor[rgb]{0.49,0.56,0.16}{##1}}}
\@namedef{PY@tok@nt}{\let\PY@bf=\textbf\def\PY@tc##1{\textcolor[rgb]{0.00,0.50,0.00}{##1}}}
\@namedef{PY@tok@nd}{\def\PY@tc##1{\textcolor[rgb]{0.67,0.13,1.00}{##1}}}
\@namedef{PY@tok@s}{\def\PY@tc##1{\textcolor[rgb]{0.73,0.13,0.13}{##1}}}
\@namedef{PY@tok@sd}{\let\PY@it=\textit\def\PY@tc##1{\textcolor[rgb]{0.73,0.13,0.13}{##1}}}
\@namedef{PY@tok@si}{\let\PY@bf=\textbf\def\PY@tc##1{\textcolor[rgb]{0.73,0.40,0.53}{##1}}}
\@namedef{PY@tok@se}{\let\PY@bf=\textbf\def\PY@tc##1{\textcolor[rgb]{0.73,0.40,0.13}{##1}}}
\@namedef{PY@tok@sr}{\def\PY@tc##1{\textcolor[rgb]{0.73,0.40,0.53}{##1}}}
\@namedef{PY@tok@ss}{\def\PY@tc##1{\textcolor[rgb]{0.10,0.09,0.49}{##1}}}
\@namedef{PY@tok@sx}{\def\PY@tc##1{\textcolor[rgb]{0.00,0.50,0.00}{##1}}}
\@namedef{PY@tok@m}{\def\PY@tc##1{\textcolor[rgb]{0.40,0.40,0.40}{##1}}}
\@namedef{PY@tok@gh}{\let\PY@bf=\textbf\def\PY@tc##1{\textcolor[rgb]{0.00,0.00,0.50}{##1}}}
\@namedef{PY@tok@gu}{\let\PY@bf=\textbf\def\PY@tc##1{\textcolor[rgb]{0.50,0.00,0.50}{##1}}}
\@namedef{PY@tok@gd}{\def\PY@tc##1{\textcolor[rgb]{0.63,0.00,0.00}{##1}}}
\@namedef{PY@tok@gi}{\def\PY@tc##1{\textcolor[rgb]{0.00,0.63,0.00}{##1}}}
\@namedef{PY@tok@gr}{\def\PY@tc##1{\textcolor[rgb]{1.00,0.00,0.00}{##1}}}
\@namedef{PY@tok@ge}{\let\PY@it=\textit}
\@namedef{PY@tok@gs}{\let\PY@bf=\textbf}
\@namedef{PY@tok@gp}{\let\PY@bf=\textbf\def\PY@tc##1{\textcolor[rgb]{0.00,0.00,0.50}{##1}}}
\@namedef{PY@tok@go}{\def\PY@tc##1{\textcolor[rgb]{0.53,0.53,0.53}{##1}}}
\@namedef{PY@tok@gt}{\def\PY@tc##1{\textcolor[rgb]{0.00,0.27,0.87}{##1}}}
\@namedef{PY@tok@err}{\def\PY@bc##1{{\setlength{\fboxsep}{\string -\fboxrule}\fcolorbox[rgb]{1.00,0.00,0.00}{1,1,1}{\strut ##1}}}}
\@namedef{PY@tok@kc}{\let\PY@bf=\textbf\def\PY@tc##1{\textcolor[rgb]{0.00,0.50,0.00}{##1}}}
\@namedef{PY@tok@kd}{\let\PY@bf=\textbf\def\PY@tc##1{\textcolor[rgb]{0.00,0.50,0.00}{##1}}}
\@namedef{PY@tok@kn}{\let\PY@bf=\textbf\def\PY@tc##1{\textcolor[rgb]{0.00,0.50,0.00}{##1}}}
\@namedef{PY@tok@kr}{\let\PY@bf=\textbf\def\PY@tc##1{\textcolor[rgb]{0.00,0.50,0.00}{##1}}}
\@namedef{PY@tok@bp}{\def\PY@tc##1{\textcolor[rgb]{0.00,0.50,0.00}{##1}}}
\@namedef{PY@tok@fm}{\def\PY@tc##1{\textcolor[rgb]{0.00,0.00,1.00}{##1}}}
\@namedef{PY@tok@vc}{\def\PY@tc##1{\textcolor[rgb]{0.10,0.09,0.49}{##1}}}
\@namedef{PY@tok@vg}{\def\PY@tc##1{\textcolor[rgb]{0.10,0.09,0.49}{##1}}}
\@namedef{PY@tok@vi}{\def\PY@tc##1{\textcolor[rgb]{0.10,0.09,0.49}{##1}}}
\@namedef{PY@tok@vm}{\def\PY@tc##1{\textcolor[rgb]{0.10,0.09,0.49}{##1}}}
\@namedef{PY@tok@sa}{\def\PY@tc##1{\textcolor[rgb]{0.73,0.13,0.13}{##1}}}
\@namedef{PY@tok@sb}{\def\PY@tc##1{\textcolor[rgb]{0.73,0.13,0.13}{##1}}}
\@namedef{PY@tok@sc}{\def\PY@tc##1{\textcolor[rgb]{0.73,0.13,0.13}{##1}}}
\@namedef{PY@tok@dl}{\def\PY@tc##1{\textcolor[rgb]{0.73,0.13,0.13}{##1}}}
\@namedef{PY@tok@s2}{\def\PY@tc##1{\textcolor[rgb]{0.73,0.13,0.13}{##1}}}
\@namedef{PY@tok@sh}{\def\PY@tc##1{\textcolor[rgb]{0.73,0.13,0.13}{##1}}}
\@namedef{PY@tok@s1}{\def\PY@tc##1{\textcolor[rgb]{0.73,0.13,0.13}{##1}}}
\@namedef{PY@tok@mb}{\def\PY@tc##1{\textcolor[rgb]{0.40,0.40,0.40}{##1}}}
\@namedef{PY@tok@mf}{\def\PY@tc##1{\textcolor[rgb]{0.40,0.40,0.40}{##1}}}
\@namedef{PY@tok@mh}{\def\PY@tc##1{\textcolor[rgb]{0.40,0.40,0.40}{##1}}}
\@namedef{PY@tok@mi}{\def\PY@tc##1{\textcolor[rgb]{0.40,0.40,0.40}{##1}}}
\@namedef{PY@tok@il}{\def\PY@tc##1{\textcolor[rgb]{0.40,0.40,0.40}{##1}}}
\@namedef{PY@tok@mo}{\def\PY@tc##1{\textcolor[rgb]{0.40,0.40,0.40}{##1}}}
\@namedef{PY@tok@ch}{\let\PY@it=\textit\def\PY@tc##1{\textcolor[rgb]{0.25,0.50,0.50}{##1}}}
\@namedef{PY@tok@cm}{\let\PY@it=\textit\def\PY@tc##1{\textcolor[rgb]{0.25,0.50,0.50}{##1}}}
\@namedef{PY@tok@cpf}{\let\PY@it=\textit\def\PY@tc##1{\textcolor[rgb]{0.25,0.50,0.50}{##1}}}
\@namedef{PY@tok@c1}{\let\PY@it=\textit\def\PY@tc##1{\textcolor[rgb]{0.25,0.50,0.50}{##1}}}
\@namedef{PY@tok@cs}{\let\PY@it=\textit\def\PY@tc##1{\textcolor[rgb]{0.25,0.50,0.50}{##1}}}

\def\PYZbs{\char`\\}
\def\PYZus{\char`\_}
\def\PYZob{\char`\{}
\def\PYZcb{\char`\}}
\def\PYZca{\char`\^}
\def\PYZam{\char`\&}
\def\PYZlt{\char`\<}
\def\PYZgt{\char`\>}
\def\PYZsh{\char`\#}
\def\PYZpc{\char`\%}
\def\PYZdl{\char`\$}
\def\PYZhy{\char`\-}
\def\PYZsq{\char`\'}
\def\PYZdq{\char`\"}
\def\PYZti{\char`\~}
% for compatibility with earlier versions
\def\PYZat{@}
\def\PYZlb{[}
\def\PYZrb{]}
\makeatother


    % For linebreaks inside Verbatim environment from package fancyvrb. 
    \makeatletter
        \newbox\Wrappedcontinuationbox 
        \newbox\Wrappedvisiblespacebox 
        \newcommand*\Wrappedvisiblespace {\textcolor{red}{\textvisiblespace}} 
        \newcommand*\Wrappedcontinuationsymbol {\textcolor{red}{\llap{\tiny$\m@th\hookrightarrow$}}} 
        \newcommand*\Wrappedcontinuationindent {3ex } 
        \newcommand*\Wrappedafterbreak {\kern\Wrappedcontinuationindent\copy\Wrappedcontinuationbox} 
        % Take advantage of the already applied Pygments mark-up to insert 
        % potential linebreaks for TeX processing. 
        %        {, <, #, %, $, ' and ": go to next line. 
        %        _, }, ^, &, >, - and ~: stay at end of broken line. 
        % Use of \textquotesingle for straight quote. 
        \newcommand*\Wrappedbreaksatspecials {% 
            \def\PYGZus{\discretionary{\char`\_}{\Wrappedafterbreak}{\char`\_}}% 
            \def\PYGZob{\discretionary{}{\Wrappedafterbreak\char`\{}{\char`\{}}% 
            \def\PYGZcb{\discretionary{\char`\}}{\Wrappedafterbreak}{\char`\}}}% 
            \def\PYGZca{\discretionary{\char`\^}{\Wrappedafterbreak}{\char`\^}}% 
            \def\PYGZam{\discretionary{\char`\&}{\Wrappedafterbreak}{\char`\&}}% 
            \def\PYGZlt{\discretionary{}{\Wrappedafterbreak\char`\<}{\char`\<}}% 
            \def\PYGZgt{\discretionary{\char`\>}{\Wrappedafterbreak}{\char`\>}}% 
            \def\PYGZsh{\discretionary{}{\Wrappedafterbreak\char`\#}{\char`\#}}% 
            \def\PYGZpc{\discretionary{}{\Wrappedafterbreak\char`\%}{\char`\%}}% 
            \def\PYGZdl{\discretionary{}{\Wrappedafterbreak\char`\$}{\char`\$}}% 
            \def\PYGZhy{\discretionary{\char`\-}{\Wrappedafterbreak}{\char`\-}}% 
            \def\PYGZsq{\discretionary{}{\Wrappedafterbreak\textquotesingle}{\textquotesingle}}% 
            \def\PYGZdq{\discretionary{}{\Wrappedafterbreak\char`\"}{\char`\"}}% 
            \def\PYGZti{\discretionary{\char`\~}{\Wrappedafterbreak}{\char`\~}}% 
        } 
        % Some characters . , ; ? ! / are not pygmentized. 
        % This macro makes them "active" and they will insert potential linebreaks 
        \newcommand*\Wrappedbreaksatpunct {% 
            \lccode`\~`\.\lowercase{\def~}{\discretionary{\hbox{\char`\.}}{\Wrappedafterbreak}{\hbox{\char`\.}}}% 
            \lccode`\~`\,\lowercase{\def~}{\discretionary{\hbox{\char`\,}}{\Wrappedafterbreak}{\hbox{\char`\,}}}% 
            \lccode`\~`\;\lowercase{\def~}{\discretionary{\hbox{\char`\;}}{\Wrappedafterbreak}{\hbox{\char`\;}}}% 
            \lccode`\~`\:\lowercase{\def~}{\discretionary{\hbox{\char`\:}}{\Wrappedafterbreak}{\hbox{\char`\:}}}% 
            \lccode`\~`\?\lowercase{\def~}{\discretionary{\hbox{\char`\?}}{\Wrappedafterbreak}{\hbox{\char`\?}}}% 
            \lccode`\~`\!\lowercase{\def~}{\discretionary{\hbox{\char`\!}}{\Wrappedafterbreak}{\hbox{\char`\!}}}% 
            \lccode`\~`\/\lowercase{\def~}{\discretionary{\hbox{\char`\/}}{\Wrappedafterbreak}{\hbox{\char`\/}}}% 
            \catcode`\.\active
            \catcode`\,\active 
            \catcode`\;\active
            \catcode`\:\active
            \catcode`\?\active
            \catcode`\!\active
            \catcode`\/\active 
            \lccode`\~`\~ 	
        }
    \makeatother

    \let\OriginalVerbatim=\Verbatim
    \makeatletter
    \renewcommand{\Verbatim}[1][1]{%
        %\parskip\z@skip
        \sbox\Wrappedcontinuationbox {\Wrappedcontinuationsymbol}%
        \sbox\Wrappedvisiblespacebox {\FV@SetupFont\Wrappedvisiblespace}%
        \def\FancyVerbFormatLine ##1{\hsize\linewidth
            \vtop{\raggedright\hyphenpenalty\z@\exhyphenpenalty\z@
                \doublehyphendemerits\z@\finalhyphendemerits\z@
                \strut ##1\strut}%
        }%
        % If the linebreak is at a space, the latter will be displayed as visible
        % space at end of first line, and a continuation symbol starts next line.
        % Stretch/shrink are however usually zero for typewriter font.
        \def\FV@Space {%
            \nobreak\hskip\z@ plus\fontdimen3\font minus\fontdimen4\font
            \discretionary{\copy\Wrappedvisiblespacebox}{\Wrappedafterbreak}
            {\kern\fontdimen2\font}%
        }%
        
        % Allow breaks at special characters using \PYG... macros.
        \Wrappedbreaksatspecials
        % Breaks at punctuation characters . , ; ? ! and / need catcode=\active 	
        \OriginalVerbatim[#1,codes*=\Wrappedbreaksatpunct]%
    }
    \makeatother

    % Exact colors from NB
    \definecolor{incolor}{HTML}{303F9F}
    \definecolor{outcolor}{HTML}{D84315}
    \definecolor{cellborder}{HTML}{CFCFCF}
    \definecolor{cellbackground}{HTML}{F7F7F7}
    
    % prompt
    \makeatletter
    \newcommand{\boxspacing}{\kern\kvtcb@left@rule\kern\kvtcb@boxsep}
    \makeatother
    \newcommand{\prompt}[4]{
        {\ttfamily\llap{{\color{#2}[#3]:\hspace{3pt}#4}}\vspace{-\baselineskip}}
    }
    

    
    % Prevent overflowing lines due to hard-to-break entities
    \sloppy 
    % Setup hyperref package
    \hypersetup{
      breaklinks=true,  % so long urls are correctly broken across lines
      colorlinks=true,
      urlcolor=urlcolor,
      linkcolor=linkcolor,
      citecolor=citecolor,
      }
    % Slightly bigger margins than the latex defaults
    
    \geometry{verbose,tmargin=1in,bmargin=1in,lmargin=1in,rmargin=1in}
    
    

\begin{document}
    
    \maketitle
    
    

    
    \begin{tcolorbox}[breakable, size=fbox, boxrule=1pt, pad at break*=1mm,colback=cellbackground, colframe=cellborder]
\prompt{In}{incolor}{2}{\boxspacing}
\begin{Verbatim}[commandchars=\\\{\}]
\PY{k+kn}{import} \PY{n+nn}{numpy} \PY{k}{as} \PY{n+nn}{np}
\PY{k+kn}{import} \PY{n+nn}{healpy} \PY{k}{as} \PY{n+nn}{hp}
\end{Verbatim}
\end{tcolorbox}

    \begin{tcolorbox}[breakable, size=fbox, boxrule=1pt, pad at break*=1mm,colback=cellbackground, colframe=cellborder]
\prompt{In}{incolor}{3}{\boxspacing}
\begin{Verbatim}[commandchars=\\\{\}]
\PY{k+kn}{import} \PY{n+nn}{matplotlib}\PY{n+nn}{.}\PY{n+nn}{pyplot} \PY{k}{as} \PY{n+nn}{plt}
\end{Verbatim}
\end{tcolorbox}

    \begin{tcolorbox}[breakable, size=fbox, boxrule=1pt, pad at break*=1mm,colback=cellbackground, colframe=cellborder]
\prompt{In}{incolor}{4}{\boxspacing}
\begin{Verbatim}[commandchars=\\\{\}]
\PY{o}{!}pwd
\end{Verbatim}
\end{tcolorbox}

    \begin{Verbatim}[commandchars=\\\{\}]
/Users/tony/project\_planckVsWMAP
    \end{Verbatim}

    \begin{tcolorbox}[breakable, size=fbox, boxrule=1pt, pad at break*=1mm,colback=cellbackground, colframe=cellborder]
\prompt{In}{incolor}{5}{\boxspacing}
\begin{Verbatim}[commandchars=\\\{\}]
\PY{n}{hp}\PY{o}{.}\PY{n}{read\PYZus{}map}\PY{p}{(}\PY{l+s+s2}{\PYZdq{}}\PY{l+s+s2}{ffp10\PYZus{}newdust\PYZus{}total\PYZus{}353\PYZus{}full\PYZus{}map.fits}\PY{l+s+s2}{\PYZdq{}}\PY{p}{)}                \PY{c+c1}{\PYZsh{}planckTmap}
\end{Verbatim}
\end{tcolorbox}

    \begin{Verbatim}[commandchars=\\\{\}]
/opt/anaconda3/envs/act\_notebooks/lib/python3.9/site-
packages/healpy/fitsfunc.py:368: UserWarning: If you are not specifying the
input dtype and using the default np.float64 dtype of read\_map(), please
consider that it will change in a future version to None as to keep the same
dtype of the input file: please explicitly set the dtype if it is important to
you.
  warnings.warn(
/opt/anaconda3/envs/act\_notebooks/lib/python3.9/site-
packages/healpy/fitsfunc.py:391: UserWarning: NSIDE = 2048
  warnings.warn("NSIDE = \{0:d\}".format(nside))
/opt/anaconda3/envs/act\_notebooks/lib/python3.9/site-
packages/healpy/fitsfunc.py:400: UserWarning: ORDERING = RING in fits file
  warnings.warn("ORDERING = \{0:s\} in fits file".format(ordering))
/opt/anaconda3/envs/act\_notebooks/lib/python3.9/site-
packages/healpy/fitsfunc.py:428: UserWarning: INDXSCHM = IMPLICIT
  warnings.warn("INDXSCHM = \{0:s\}".format(schm))
    \end{Verbatim}

            \begin{tcolorbox}[breakable, size=fbox, boxrule=.5pt, pad at break*=1mm, opacityfill=0]
\prompt{Out}{outcolor}{5}{\boxspacing}
\begin{Verbatim}[commandchars=\\\{\}]
array([ 2.50327867e-06, -4.45016194e-05,  3.31364572e-05, {\ldots},
       -4.48527280e-05, -1.91247091e-05,  2.42725946e-05])
\end{Verbatim}
\end{tcolorbox}
        
    \begin{tcolorbox}[breakable, size=fbox, boxrule=1pt, pad at break*=1mm,colback=cellbackground, colframe=cellborder]
\prompt{In}{incolor}{6}{\boxspacing}
\begin{Verbatim}[commandchars=\\\{\}]
\PY{n}{map\PYZus{}sample} \PY{o}{=} \PY{n}{hp}\PY{o}{.}\PY{n}{read\PYZus{}map}\PY{p}{(}\PY{l+s+s2}{\PYZdq{}}\PY{l+s+s2}{ffp10\PYZus{}newdust\PYZus{}total\PYZus{}030\PYZus{}full\PYZus{}map.fits}\PY{l+s+s2}{\PYZdq{}}\PY{p}{)}
\end{Verbatim}
\end{tcolorbox}

    \begin{Verbatim}[commandchars=\\\{\}]
/opt/anaconda3/envs/act\_notebooks/lib/python3.9/site-
packages/healpy/fitsfunc.py:391: UserWarning: NSIDE = 1024
  warnings.warn("NSIDE = \{0:d\}".format(nside))
/opt/anaconda3/envs/act\_notebooks/lib/python3.9/site-
packages/healpy/fitsfunc.py:400: UserWarning: ORDERING = NESTED in fits file
  warnings.warn("ORDERING = \{0:s\} in fits file".format(ordering))
/opt/anaconda3/envs/act\_notebooks/lib/python3.9/site-
packages/healpy/fitsfunc.py:486: UserWarning: Ordering converted to RING
  warnings.warn("Ordering converted to RING")
    \end{Verbatim}

    \begin{tcolorbox}[breakable, size=fbox, boxrule=1pt, pad at break*=1mm,colback=cellbackground, colframe=cellborder]
\prompt{In}{incolor}{ }{\boxspacing}
\begin{Verbatim}[commandchars=\\\{\}]

\end{Verbatim}
\end{tcolorbox}

    \begin{tcolorbox}[breakable, size=fbox, boxrule=1pt, pad at break*=1mm,colback=cellbackground, colframe=cellborder]
\prompt{In}{incolor}{7}{\boxspacing}
\begin{Verbatim}[commandchars=\\\{\}]
\PY{n}{map\PYZus{}sample}    \PY{c+c1}{\PYZsh{}353Hz map}
\end{Verbatim}
\end{tcolorbox}

            \begin{tcolorbox}[breakable, size=fbox, boxrule=.5pt, pad at break*=1mm, opacityfill=0]
\prompt{Out}{outcolor}{7}{\boxspacing}
\begin{Verbatim}[commandchars=\\\{\}]
array([-2.71337165e-04, -9.36984798e-05, -2.51075049e-04, {\ldots},
       -1.52736000e-04, -1.10948109e-04, -7.72219792e-05])
\end{Verbatim}
\end{tcolorbox}
        
    \begin{tcolorbox}[breakable, size=fbox, boxrule=1pt, pad at break*=1mm,colback=cellbackground, colframe=cellborder]
\prompt{In}{incolor}{8}{\boxspacing}
\begin{Verbatim}[commandchars=\\\{\}]
\PY{n}{hp}\PY{o}{.}\PY{n}{mollview}\PY{p}{(}
    \PY{n}{map\PYZus{}sample}\PY{p}{,}
    \PY{n}{coord}\PY{o}{=}\PY{p}{[}\PY{l+s+s2}{\PYZdq{}}\PY{l+s+s2}{G}\PY{l+s+s2}{\PYZdq{}}\PY{p}{,} \PY{l+s+s2}{\PYZdq{}}\PY{l+s+s2}{E}\PY{l+s+s2}{\PYZdq{}}\PY{p}{]}\PY{p}{,}
    \PY{n}{title}\PY{o}{=}\PY{l+s+s2}{\PYZdq{}}\PY{l+s+s2}{Histogram equalized Ecliptic Sample 30Hz}\PY{l+s+s2}{\PYZdq{}}\PY{p}{,}
    \PY{n}{unit}\PY{o}{=}\PY{l+s+s2}{\PYZdq{}}\PY{l+s+s2}{mK}\PY{l+s+s2}{\PYZdq{}}\PY{p}{,}
    \PY{n}{norm}\PY{o}{=}\PY{l+s+s2}{\PYZdq{}}\PY{l+s+s2}{hist}\PY{l+s+s2}{\PYZdq{}}\PY{p}{,}
    \PY{n+nb}{min}\PY{o}{=}\PY{o}{\PYZhy{}}\PY{l+m+mf}{0.01}\PY{p}{,}
    \PY{n+nb}{max}\PY{o}{=}\PY{l+m+mf}{0.01}\PY{p}{,}
\PY{p}{)}
\PY{n}{hp}\PY{o}{.}\PY{n}{graticule}\PY{p}{(}\PY{p}{)}
\end{Verbatim}
\end{tcolorbox}

    \begin{Verbatim}[commandchars=\\\{\}]
/opt/anaconda3/envs/act\_notebooks/lib/python3.9/site-
packages/healpy/projaxes.py:920: MatplotlibDeprecationWarning: You are modifying
the state of a globally registered colormap. This has been deprecated since 3.3
and in 3.6, you will not be able to modify a registered colormap in-place. To
remove this warning, you can make a copy of the colormap first. cmap =
mpl.cm.get\_cmap("viridis").copy()
  newcm.set\_over(newcm(1.0))
/opt/anaconda3/envs/act\_notebooks/lib/python3.9/site-
packages/healpy/projaxes.py:921: MatplotlibDeprecationWarning: You are modifying
the state of a globally registered colormap. This has been deprecated since 3.3
and in 3.6, you will not be able to modify a registered colormap in-place. To
remove this warning, you can make a copy of the colormap first. cmap =
mpl.cm.get\_cmap("viridis").copy()
  newcm.set\_under(bgcolor)
/opt/anaconda3/envs/act\_notebooks/lib/python3.9/site-
packages/healpy/projaxes.py:922: MatplotlibDeprecationWarning: You are modifying
the state of a globally registered colormap. This has been deprecated since 3.3
and in 3.6, you will not be able to modify a registered colormap in-place. To
remove this warning, you can make a copy of the colormap first. cmap =
mpl.cm.get\_cmap("viridis").copy()
  newcm.set\_bad(badcolor)
/opt/anaconda3/envs/act\_notebooks/lib/python3.9/site-
packages/healpy/projaxes.py:202: MatplotlibDeprecationWarning: Passing
parameters norm and vmin/vmax simultaneously is deprecated since 3.3 and will
become an error two minor releases later. Please pass vmin/vmax directly to the
norm when creating it.
  aximg = self.imshow(
/opt/anaconda3/envs/act\_notebooks/lib/python3.9/site-
packages/healpy/projaxes.py:541: UserWarning: 0.0 180.0 -180.0 180.0
  warnings.warn(
/opt/anaconda3/envs/act\_notebooks/lib/python3.9/site-
packages/healpy/projaxes.py:656: UserWarning: The interval between parallels is
30 deg -0.00'.
  warnings.warn(
/opt/anaconda3/envs/act\_notebooks/lib/python3.9/site-
packages/healpy/projaxes.py:664: UserWarning: The interval between meridians is
30 deg -0.00'.
  warnings.warn(
    \end{Verbatim}

    \begin{center}
    \adjustimage{max size={0.9\linewidth}{0.9\paperheight}}{output_7_1.png}
    \end{center}
    { \hspace*{\fill} \\}
    
    Beam Maps

    \begin{tcolorbox}[breakable, size=fbox, boxrule=1pt, pad at break*=1mm,colback=cellbackground, colframe=cellborder]
\prompt{In}{incolor}{9}{\boxspacing}
\begin{Verbatim}[commandchars=\\\{\}]
\PY{k+kn}{import} \PY{n+nn}{astropy}\PY{n+nn}{.}\PY{n+nn}{io}\PY{n+nn}{.}\PY{n+nn}{fits} \PY{k}{as} \PY{n+nn}{pf}
\PY{k+kn}{import} \PY{n+nn}{pylab} \PY{c+c1}{\PYZsh{} only to produce plots}
\end{Verbatim}
\end{tcolorbox}

    \begin{tcolorbox}[breakable, size=fbox, boxrule=1pt, pad at break*=1mm,colback=cellbackground, colframe=cellborder]
\prompt{In}{incolor}{10}{\boxspacing}
\begin{Verbatim}[commandchars=\\\{\}]
\PY{n}{FITSfile} \PY{o}{=} \PY{l+s+s1}{\PYZsq{}}\PY{l+s+s1}{BeamWf\PYZus{}HFI\PYZus{}R3.01/Bl\PYZus{}T\PYZus{}R3.01\PYZus{}fullsky\PYZus{}353x353.fits}\PY{l+s+s1}{\PYZsq{}}

\PY{n}{pf}\PY{o}{.}\PY{n}{info}\PY{p}{(}\PY{n}{FITSfile}\PY{p}{)} \PY{c+c1}{\PYZsh{} print list of extensions found in FITSfile}
\PY{n}{data}\PY{p}{,} \PY{n}{header} \PY{o}{=} \PY{n}{pf}\PY{o}{.}\PY{n}{getdata}\PY{p}{(}\PY{n}{FITSfile}\PY{p}{,} \PY{n}{header}\PY{o}{=}\PY{k+kc}{True}\PY{p}{)}\PY{c+c1}{\PYZsh{}, 10, header=True) \PYZsh{} read extension \PYZsh{}10 (data and header)}
\PY{c+c1}{\PYZsh{}data, header = pf.getdata(FITSfile100, \PYZsq{}100\PYZsq{}, header=True) \PYZsh{} read extension having EXTNAME=\PYZsq{}ABC\PYZsq{} (data and header)}
\PY{n+nb}{print}\PY{p}{(}\PY{n}{header}\PY{p}{)} \PY{c+c1}{\PYZsh{} print header}
\PY{n+nb}{print}\PY{p}{(}\PY{n}{data}\PY{o}{.}\PY{n}{names}\PY{p}{)} \PY{c+c1}{\PYZsh{} print column names}
\PY{n}{pylab}\PY{o}{.}\PY{n}{plot}\PY{p}{(} \PY{n}{data}\PY{o}{.}\PY{n}{field}\PY{p}{(}\PY{l+m+mi}{0}\PY{p}{)}\PY{o}{.}\PY{n}{flatten}\PY{p}{(}\PY{p}{)} \PY{p}{)} \PY{c+c1}{\PYZsh{} plot 1st column of binary table}
\end{Verbatim}
\end{tcolorbox}

    \begin{Verbatim}[commandchars=\\\{\}]
Filename: BeamWf\_HFI\_R3.01/Bl\_T\_R3.01\_fullsky\_353x353.fits
No.    Name      Ver    Type      Cards   Dimensions   Format
  0  PRIMARY       1 PrimaryHDU      19   ()
  1  WINDOW FUNCTION    1 TableHDU        26   4001R x 1C   [E15.7]
XTENSION= 'TABLE   '           / ASCII table extension
BITPIX  =                    8 / array data type
NAXIS   =                    2 / number of array dimensions
NAXIS1  =                   15 / length of dimension 1
NAXIS2  =                 4001 / length of dimension 2
PCOUNT  =                    0 / number of group parameters
GCOUNT  =                    1 / number of groups
TFIELDS =                    1 / number of table fields
TTYPE1  = 'TEMPERATURE'
TFORM1  = 'E15.7   '
TBCOL1  =                    1
EXTNAME = 'WINDOW FUNCTION'    / extension name
MAX-LPOL=                 4000 / Maximum L multipole
POLAR   =                    F
BCROSS  =                    F
ASYMCL  =                    F
COMMENT ----------------------------------------------------------------
COMMENT Beam Window Function B(l)
COMMENT Compatible with Healpix (synfast, smoothing, {\ldots}) and PolSpice
COMMENT To be squared before applying to power spectrum
COMMENT   C\_map(l) = C\_sky(l) * B(l)\^{}2
COMMENT Adapted from
COMMENT
/redtruck/?????/m3space/quickbeam/quickpol/data/RD12rc3/beam\_matrix\_353xCOMMENT
353\_l4000\_s6\_A000\_cmbfast\_0\_rbIMO\_rhIMO.npz                             COMMENT
by ./qp2fits.py on 2016-06-15                                           COMMENT
----------------------------------------------------------------        END
['TEMPERATURE']
    \end{Verbatim}

            \begin{tcolorbox}[breakable, size=fbox, boxrule=.5pt, pad at break*=1mm, opacityfill=0]
\prompt{Out}{outcolor}{10}{\boxspacing}
\begin{Verbatim}[commandchars=\\\{\}]
[<matplotlib.lines.Line2D at 0x102eca520>]
\end{Verbatim}
\end{tcolorbox}
        
    \begin{center}
    \adjustimage{max size={0.9\linewidth}{0.9\paperheight}}{output_10_2.png}
    \end{center}
    { \hspace*{\fill} \\}
    
    \begin{tcolorbox}[breakable, size=fbox, boxrule=1pt, pad at break*=1mm,colback=cellbackground, colframe=cellborder]
\prompt{In}{incolor}{11}{\boxspacing}
\begin{Verbatim}[commandchars=\\\{\}]
\PY{n}{data}
\end{Verbatim}
\end{tcolorbox}

            \begin{tcolorbox}[breakable, size=fbox, boxrule=.5pt, pad at break*=1mm, opacityfill=0]
\prompt{Out}{outcolor}{11}{\boxspacing}
\begin{Verbatim}[commandchars=\\\{\}]
FITS\_rec([(1.0,), (0.99999952,), (0.99999863,), {\ldots}, (0.068820141,),
          (0.068731263,), (0.068642475,)],
         dtype=(numpy.record, [('TEMPERATURE', 'S15')]))
\end{Verbatim}
\end{tcolorbox}
        
    \begin{tcolorbox}[breakable, size=fbox, boxrule=1pt, pad at break*=1mm,colback=cellbackground, colframe=cellborder]
\prompt{In}{incolor}{12}{\boxspacing}
\begin{Verbatim}[commandchars=\\\{\}]
\PY{c+c1}{\PYZsh{}Intensity Map (Planck) \PYZhy{} Foreground removed}


\PY{n}{intensityCMB100} \PY{o}{=} \PY{n}{hp}\PY{o}{.}\PY{n}{read\PYZus{}map}\PY{p}{(}\PY{l+s+s2}{\PYZdq{}}\PY{l+s+s2}{COM\PYZus{}CMB\PYZus{}IQU\PYZhy{}100\PYZhy{}fgsub\PYZhy{}sevem\PYZhy{}field\PYZhy{}Int\PYZus{}2048\PYZus{}R2.01\PYZus{}full.fits}\PY{l+s+s2}{\PYZdq{}}\PY{p}{)}
\PY{n}{intensityCMB100}
\end{Verbatim}
\end{tcolorbox}

    \begin{Verbatim}[commandchars=\\\{\}]
/opt/anaconda3/envs/act\_notebooks/lib/python3.9/site-
packages/healpy/fitsfunc.py:368: UserWarning: If you are not specifying the
input dtype and using the default np.float64 dtype of read\_map(), please
consider that it will change in a future version to None as to keep the same
dtype of the input file: please explicitly set the dtype if it is important to
you.
  warnings.warn(
/opt/anaconda3/envs/act\_notebooks/lib/python3.9/site-
packages/healpy/fitsfunc.py:391: UserWarning: NSIDE = 2048
  warnings.warn("NSIDE = \{0:d\}".format(nside))
/opt/anaconda3/envs/act\_notebooks/lib/python3.9/site-
packages/healpy/fitsfunc.py:400: UserWarning: ORDERING = NESTED in fits file
  warnings.warn("ORDERING = \{0:s\} in fits file".format(ordering))
/opt/anaconda3/envs/act\_notebooks/lib/python3.9/site-
packages/healpy/fitsfunc.py:426: UserWarning: No INDXSCHM keyword in header file
: assume IMPLICIT
  warnings.warn("No INDXSCHM keyword in header file : " "assume
\{\}".format(schm))
/opt/anaconda3/envs/act\_notebooks/lib/python3.9/site-
packages/healpy/fitsfunc.py:428: UserWarning: INDXSCHM = IMPLICIT
  warnings.warn("INDXSCHM = \{0:s\}".format(schm))
/opt/anaconda3/envs/act\_notebooks/lib/python3.9/site-
packages/healpy/fitsfunc.py:486: UserWarning: Ordering converted to RING
  warnings.warn("Ordering converted to RING")
    \end{Verbatim}

            \begin{tcolorbox}[breakable, size=fbox, boxrule=.5pt, pad at break*=1mm, opacityfill=0]
\prompt{Out}{outcolor}{12}{\boxspacing}
\begin{Verbatim}[commandchars=\\\{\}]
array([-1.34818620e-04, -1.13565744e-04, -5.30425641e-05, {\ldots},
        9.37869918e-05,  3.33317512e-05, -2.04297194e-05])
\end{Verbatim}
\end{tcolorbox}
        
    \begin{tcolorbox}[breakable, size=fbox, boxrule=1pt, pad at break*=1mm,colback=cellbackground, colframe=cellborder]
\prompt{In}{incolor}{13}{\boxspacing}
\begin{Verbatim}[commandchars=\\\{\}]
\PY{n}{hp}\PY{o}{.}\PY{n}{mollview}\PY{p}{(}
    \PY{n}{intensityCMB100}\PY{p}{,}
    \PY{n}{coord}\PY{o}{=}\PY{p}{[}\PY{l+s+s2}{\PYZdq{}}\PY{l+s+s2}{G}\PY{l+s+s2}{\PYZdq{}}\PY{p}{,} \PY{l+s+s2}{\PYZdq{}}\PY{l+s+s2}{E}\PY{l+s+s2}{\PYZdq{}}\PY{p}{]}\PY{p}{,}
    \PY{n}{title}\PY{o}{=}\PY{l+s+s2}{\PYZdq{}}\PY{l+s+s2}{Intensity CMB}\PY{l+s+s2}{\PYZdq{}}\PY{p}{,}
    \PY{n}{unit}\PY{o}{=}\PY{l+s+s2}{\PYZdq{}}\PY{l+s+s2}{mK}\PY{l+s+s2}{\PYZdq{}}\PY{p}{,}
    \PY{n}{norm}\PY{o}{=}\PY{l+s+s2}{\PYZdq{}}\PY{l+s+s2}{hist}\PY{l+s+s2}{\PYZdq{}}\PY{p}{,}
    \PY{n+nb}{min}\PY{o}{=}\PY{o}{\PYZhy{}}\PY{l+m+mf}{0.01}\PY{p}{,}
    \PY{n+nb}{max}\PY{o}{=}\PY{l+m+mf}{0.01}\PY{p}{,}
\PY{p}{)}
\PY{n}{hp}\PY{o}{.}\PY{n}{graticule}\PY{p}{(}\PY{p}{)}
\end{Verbatim}
\end{tcolorbox}

    \begin{Verbatim}[commandchars=\\\{\}]
/opt/anaconda3/envs/act\_notebooks/lib/python3.9/site-
packages/healpy/projaxes.py:920: MatplotlibDeprecationWarning: You are modifying
the state of a globally registered colormap. This has been deprecated since 3.3
and in 3.6, you will not be able to modify a registered colormap in-place. To
remove this warning, you can make a copy of the colormap first. cmap =
mpl.cm.get\_cmap("viridis").copy()
  newcm.set\_over(newcm(1.0))
/opt/anaconda3/envs/act\_notebooks/lib/python3.9/site-
packages/healpy/projaxes.py:921: MatplotlibDeprecationWarning: You are modifying
the state of a globally registered colormap. This has been deprecated since 3.3
and in 3.6, you will not be able to modify a registered colormap in-place. To
remove this warning, you can make a copy of the colormap first. cmap =
mpl.cm.get\_cmap("viridis").copy()
  newcm.set\_under(bgcolor)
/opt/anaconda3/envs/act\_notebooks/lib/python3.9/site-
packages/healpy/projaxes.py:922: MatplotlibDeprecationWarning: You are modifying
the state of a globally registered colormap. This has been deprecated since 3.3
and in 3.6, you will not be able to modify a registered colormap in-place. To
remove this warning, you can make a copy of the colormap first. cmap =
mpl.cm.get\_cmap("viridis").copy()
  newcm.set\_bad(badcolor)
/opt/anaconda3/envs/act\_notebooks/lib/python3.9/site-
packages/healpy/projaxes.py:202: MatplotlibDeprecationWarning: Passing
parameters norm and vmin/vmax simultaneously is deprecated since 3.3 and will
become an error two minor releases later. Please pass vmin/vmax directly to the
norm when creating it.
  aximg = self.imshow(
/opt/anaconda3/envs/act\_notebooks/lib/python3.9/site-
packages/healpy/projaxes.py:541: UserWarning: 0.0 180.0 -180.0 180.0
  warnings.warn(
/opt/anaconda3/envs/act\_notebooks/lib/python3.9/site-
packages/healpy/projaxes.py:656: UserWarning: The interval between parallels is
30 deg -0.00'.
  warnings.warn(
/opt/anaconda3/envs/act\_notebooks/lib/python3.9/site-
packages/healpy/projaxes.py:664: UserWarning: The interval between meridians is
30 deg -0.00'.
  warnings.warn(
    \end{Verbatim}

    \begin{center}
    \adjustimage{max size={0.9\linewidth}{0.9\paperheight}}{output_13_1.png}
    \end{center}
    { \hspace*{\fill} \\}
    
    \begin{tcolorbox}[breakable, size=fbox, boxrule=1pt, pad at break*=1mm,colback=cellbackground, colframe=cellborder]
\prompt{In}{incolor}{14}{\boxspacing}
\begin{Verbatim}[commandchars=\\\{\}]
\PY{n}{intensityCMB217} \PY{o}{=} \PY{n}{hp}\PY{o}{.}\PY{n}{read\PYZus{}map}\PY{p}{(}\PY{l+s+s2}{\PYZdq{}}\PY{l+s+s2}{COM\PYZus{}CMB\PYZus{}IQU\PYZhy{}217\PYZhy{}fgsub\PYZhy{}sevem\PYZhy{}field\PYZhy{}Int\PYZus{}2048\PYZus{}R2.01\PYZus{}full.fits}\PY{l+s+s2}{\PYZdq{}}\PY{p}{)}
\PY{n}{intensityCMB217}
\end{Verbatim}
\end{tcolorbox}

            \begin{tcolorbox}[breakable, size=fbox, boxrule=.5pt, pad at break*=1mm, opacityfill=0]
\prompt{Out}{outcolor}{14}{\boxspacing}
\begin{Verbatim}[commandchars=\\\{\}]
array([-1.45112805e-04, -9.79054021e-05, -8.30674253e-05, {\ldots},
        1.03153470e-04,  9.47897715e-05,  8.28015836e-05])
\end{Verbatim}
\end{tcolorbox}
        
    \begin{tcolorbox}[breakable, size=fbox, boxrule=1pt, pad at break*=1mm,colback=cellbackground, colframe=cellborder]
\prompt{In}{incolor}{15}{\boxspacing}
\begin{Verbatim}[commandchars=\\\{\}]
\PY{n}{hp}\PY{o}{.}\PY{n}{mollview}\PY{p}{(}
    \PY{n}{intensityCMB217}\PY{p}{,}
    \PY{n}{coord}\PY{o}{=}\PY{p}{[}\PY{l+s+s2}{\PYZdq{}}\PY{l+s+s2}{G}\PY{l+s+s2}{\PYZdq{}}\PY{p}{,} \PY{l+s+s2}{\PYZdq{}}\PY{l+s+s2}{E}\PY{l+s+s2}{\PYZdq{}}\PY{p}{]}\PY{p}{,}
    \PY{n}{title}\PY{o}{=}\PY{l+s+s2}{\PYZdq{}}\PY{l+s+s2}{Intensity CMB}\PY{l+s+s2}{\PYZdq{}}\PY{p}{,}
    \PY{n}{unit}\PY{o}{=}\PY{l+s+s2}{\PYZdq{}}\PY{l+s+s2}{mK}\PY{l+s+s2}{\PYZdq{}}\PY{p}{,}
    \PY{n}{norm}\PY{o}{=}\PY{l+s+s2}{\PYZdq{}}\PY{l+s+s2}{hist}\PY{l+s+s2}{\PYZdq{}}\PY{p}{,}
    \PY{n+nb}{min}\PY{o}{=}\PY{o}{\PYZhy{}}\PY{l+m+mf}{0.01}\PY{p}{,}
    \PY{n+nb}{max}\PY{o}{=}\PY{l+m+mf}{0.01}\PY{p}{,}
\PY{p}{)}
\PY{n}{hp}\PY{o}{.}\PY{n}{graticule}\PY{p}{(}\PY{p}{)}
\end{Verbatim}
\end{tcolorbox}

    \begin{center}
    \adjustimage{max size={0.9\linewidth}{0.9\paperheight}}{output_15_0.png}
    \end{center}
    { \hspace*{\fill} \\}
    
    WMAP

    \begin{tcolorbox}[breakable, size=fbox, boxrule=1pt, pad at break*=1mm,colback=cellbackground, colframe=cellborder]
\prompt{In}{incolor}{16}{\boxspacing}
\begin{Verbatim}[commandchars=\\\{\}]
\PY{o}{!}pwd
\end{Verbatim}
\end{tcolorbox}

    \begin{Verbatim}[commandchars=\\\{\}]
/Users/tony/project\_planckVsWMAP
    \end{Verbatim}

    \hypertarget{computing-the-planck-cmb-temperature-power-spectrum-with-healpy-anafast}{%
\subsubsection{Computing the Planck CMB temperature power spectrum with
healpy
anafast}\label{computing-the-planck-cmb-temperature-power-spectrum-with-healpy-anafast}}

    \begin{tcolorbox}[breakable, size=fbox, boxrule=1pt, pad at break*=1mm,colback=cellbackground, colframe=cellborder]
\prompt{In}{incolor}{17}{\boxspacing}
\begin{Verbatim}[commandchars=\\\{\}]
\PY{k+kn}{import} \PY{n+nn}{healpy} \PY{k}{as} \PY{n+nn}{hp}
\PY{k+kn}{import} \PY{n+nn}{numpy} \PY{k}{as} \PY{n+nn}{np}
\PY{k+kn}{import} \PY{n+nn}{os}
\PY{k+kn}{import} \PY{n+nn}{astropy}\PY{n+nn}{.}\PY{n+nn}{units} \PY{k}{as} \PY{n+nn}{u}
\PY{k+kn}{import} \PY{n+nn}{matplotlib}\PY{n+nn}{.}\PY{n+nn}{pyplot} \PY{k}{as} \PY{n+nn}{plt}
\PY{o}{\PYZpc{}}\PY{k}{matplotlib} inline
\end{Verbatim}
\end{tcolorbox}

    \#step 1: First we load the Planck data release 3 CMB-only temperature
produced by Commander by separating it out from galactic foregrounds:

    .

    Tool used

!curl
``https://irsa.ipac.caltech.edu/data/Planck/release\_3/all-sky-maps/maps/component-maps/cmb/COM\_CMB\_IQU-commander\_2048\_R3.00\_full.fits''
-o COM\_CMB\_IQU-commander\_2048\_R3.00\_full.fits

    \begin{tcolorbox}[breakable, size=fbox, boxrule=1pt, pad at break*=1mm,colback=cellbackground, colframe=cellborder]
\prompt{In}{incolor}{18}{\boxspacing}
\begin{Verbatim}[commandchars=\\\{\}]
\PY{n}{filename} \PY{o}{=} \PY{l+s+s1}{\PYZsq{}}\PY{l+s+s1}{COM\PYZus{}CMB\PYZus{}IQU\PYZhy{}commander\PYZus{}2048\PYZus{}R3.00\PYZus{}full.fits}\PY{l+s+s1}{\PYZsq{}}
\PY{n}{cmb\PYZus{}map} \PY{o}{=} \PY{n}{hp}\PY{o}{.}\PY{n}{read\PYZus{}map}\PY{p}{(}\PY{n}{filename}\PY{p}{)}
\end{Verbatim}
\end{tcolorbox}

    \#loading masks file

    Tool used

!curl
``https://irsa.ipac.caltech.edu/data/Planck/release\_3/ancillary-data/masks/COM\_Mask\_CMB-common-Mask-Int\_2048\_R3.00.fits''
-o COM\_Mask\_CMB-common-Mask-Int\_2048\_R3.00.fits

    \begin{tcolorbox}[breakable, size=fbox, boxrule=1pt, pad at break*=1mm,colback=cellbackground, colframe=cellborder]
\prompt{In}{incolor}{19}{\boxspacing}
\begin{Verbatim}[commandchars=\\\{\}]
\PY{c+c1}{\PYZsh{}Visualising cmb\PYZus{}map}
\PY{n}{hp}\PY{o}{.}\PY{n}{mollview}\PY{p}{(}\PY{n}{cmb\PYZus{}map}\PY{p}{,} \PY{n+nb}{min}\PY{o}{=}\PY{o}{\PYZhy{}}\PY{l+m+mf}{1e\PYZhy{}3}\PY{p}{,} \PY{n+nb}{max}\PY{o}{=}\PY{l+m+mf}{1e\PYZhy{}3}\PY{p}{,} \PY{n}{title}\PY{o}{=}\PY{l+s+s2}{\PYZdq{}}\PY{l+s+s2}{CMB only temperature map}\PY{l+s+s2}{\PYZdq{}}\PY{p}{,} \PY{n}{unit}\PY{o}{=}\PY{l+s+s2}{\PYZdq{}}\PY{l+s+s2}{K}\PY{l+s+s2}{\PYZdq{}}\PY{p}{)}
\end{Verbatim}
\end{tcolorbox}

    \begin{center}
    \adjustimage{max size={0.9\linewidth}{0.9\paperheight}}{output_26_0.png}
    \end{center}
    { \hspace*{\fill} \\}
    
    Observation: There is residual galactic emission we should mask. The
contamination just close to the galactic plane, (we could run anafast
and specify a few degrees of gal\_cut).

    \begin{tcolorbox}[breakable, size=fbox, boxrule=1pt, pad at break*=1mm,colback=cellbackground, colframe=cellborder]
\prompt{In}{incolor}{20}{\boxspacing}
\begin{Verbatim}[commandchars=\\\{\}]
\PY{c+c1}{\PYZsh{}masking with one of the planck masks(earlier loaded)}
\PY{n}{path} \PY{o}{=} \PY{l+s+s1}{\PYZsq{}}\PY{l+s+s1}{COM\PYZus{}Mask\PYZus{}CMB\PYZhy{}common\PYZhy{}Mask\PYZhy{}Int\PYZus{}2048\PYZus{}R3.00.fits}\PY{l+s+s1}{\PYZsq{}}
\PY{n}{mask} \PY{o}{=} \PY{n}{hp}\PY{o}{.}\PY{n}{read\PYZus{}map}\PY{p}{(}\PY{n}{path}\PY{p}{)}
\PY{n}{map\PYZus{}masked} \PY{o}{=} \PY{n}{hp}\PY{o}{.}\PY{n}{ma}\PY{p}{(}\PY{n}{cmb\PYZus{}map}\PY{p}{)}
\PY{n}{map\PYZus{}masked}\PY{o}{.}\PY{n}{mask} \PY{o}{=} \PY{n}{np}\PY{o}{.}\PY{n}{logical\PYZus{}not}\PY{p}{(}\PY{n}{mask}\PY{p}{)}
\end{Verbatim}
\end{tcolorbox}

    \begin{tcolorbox}[breakable, size=fbox, boxrule=1pt, pad at break*=1mm,colback=cellbackground, colframe=cellborder]
\prompt{In}{incolor}{21}{\boxspacing}
\begin{Verbatim}[commandchars=\\\{\}]
\PY{n}{mask}
\end{Verbatim}
\end{tcolorbox}

            \begin{tcolorbox}[breakable, size=fbox, boxrule=.5pt, pad at break*=1mm, opacityfill=0]
\prompt{Out}{outcolor}{21}{\boxspacing}
\begin{Verbatim}[commandchars=\\\{\}]
array([1., 1., 1., {\ldots}, 1., 1., 1.])
\end{Verbatim}
\end{tcolorbox}
        
    \begin{tcolorbox}[breakable, size=fbox, boxrule=1pt, pad at break*=1mm,colback=cellbackground, colframe=cellborder]
\prompt{In}{incolor}{22}{\boxspacing}
\begin{Verbatim}[commandchars=\\\{\}]
\PY{n}{map\PYZus{}masked}
\end{Verbatim}
\end{tcolorbox}

            \begin{tcolorbox}[breakable, size=fbox, boxrule=.5pt, pad at break*=1mm, opacityfill=0]
\prompt{Out}{outcolor}{22}{\boxspacing}
\begin{Verbatim}[commandchars=\\\{\}]
masked\_array(data=[-0.00014912881306372583, -0.00011455055937403813,
                   -9.056342241819948e-05, {\ldots}, 0.00011307044769637287,
                   9.498728468315676e-05, 0.00010467269021319225],
             mask=[False, False, False, {\ldots}, False, False, False],
       fill\_value=-1.6375e+30)
\end{Verbatim}
\end{tcolorbox}
        
    \begin{tcolorbox}[breakable, size=fbox, boxrule=1pt, pad at break*=1mm,colback=cellbackground, colframe=cellborder]
\prompt{In}{incolor}{23}{\boxspacing}
\begin{Verbatim}[commandchars=\\\{\}]
\PY{n}{map\PYZus{}masked}\PY{o}{.}\PY{n}{mask}
\end{Verbatim}
\end{tcolorbox}

            \begin{tcolorbox}[breakable, size=fbox, boxrule=.5pt, pad at break*=1mm, opacityfill=0]
\prompt{Out}{outcolor}{23}{\boxspacing}
\begin{Verbatim}[commandchars=\\\{\}]
array([False, False, False, {\ldots}, False, False, False])
\end{Verbatim}
\end{tcolorbox}
        
    \begin{tcolorbox}[breakable, size=fbox, boxrule=1pt, pad at break*=1mm,colback=cellbackground, colframe=cellborder]
\prompt{In}{incolor}{24}{\boxspacing}
\begin{Verbatim}[commandchars=\\\{\}]
\PY{n}{hp}\PY{o}{.}\PY{n}{mollview}\PY{p}{(}\PY{n}{map\PYZus{}masked}\PY{p}{,} \PY{n+nb}{min}\PY{o}{=}\PY{o}{\PYZhy{}}\PY{l+m+mf}{1e\PYZhy{}3}\PY{p}{,} \PY{n+nb}{max}\PY{o}{=}\PY{l+m+mf}{1e\PYZhy{}3}\PY{p}{)}
\end{Verbatim}
\end{tcolorbox}

    \begin{center}
    \adjustimage{max size={0.9\linewidth}{0.9\paperheight}}{output_32_0.png}
    \end{center}
    { \hspace*{\fill} \\}
    
    Now,we can load the binned TT CMB power spectrum:

    Tool used

!curl
``https://irsa.ipac.caltech.edu/data/Planck/release\_3/ancillary-data/cosmoparams/COM\_PowerSpect\_CMB-TT-binned\_R3.01.txt''
-o COM\_PowerSpect\_CMB-TT-binned\_R3.01.txt

    \begin{tcolorbox}[breakable, size=fbox, boxrule=1pt, pad at break*=1mm,colback=cellbackground, colframe=cellborder]
\prompt{In}{incolor}{25}{\boxspacing}
\begin{Verbatim}[commandchars=\\\{\}]
\PY{o}{!}head \PYZhy{}3 COM\PYZus{}PowerSpect\PYZus{}CMB\PYZhy{}TT\PYZhy{}binned\PYZus{}R3.01.txt          \PYZsh{}use of header info?
\end{Verbatim}
\end{tcolorbox}

    \begin{Verbatim}[commandchars=\\\{\}]
\# l                Dl               -dDl             +dDl             BestFit
  4.77112240e+01   1.47933552e+03   5.07654876e+01   5.07654876e+01
1.46111304e+03
  7.64716065e+01   2.03496833e+03   5.47101576e+01   5.47101576e+01
2.06238073e+03
    \end{Verbatim}

    \begin{tcolorbox}[breakable, size=fbox, boxrule=1pt, pad at break*=1mm,colback=cellbackground, colframe=cellborder]
\prompt{In}{incolor}{26}{\boxspacing}
\begin{Verbatim}[commandchars=\\\{\}]
\PY{n}{cmb\PYZus{}binned\PYZus{}spectrum} \PY{o}{=} \PY{n}{np}\PY{o}{.}\PY{n}{loadtxt}\PY{p}{(}\PY{l+s+s1}{\PYZsq{}}\PY{l+s+s1}{COM\PYZus{}PowerSpect\PYZus{}CMB\PYZhy{}TT\PYZhy{}binned\PYZus{}R3.01.txt}\PY{l+s+s1}{\PYZsq{}}\PY{p}{)}
\end{Verbatim}
\end{tcolorbox}

    \begin{tcolorbox}[breakable, size=fbox, boxrule=1pt, pad at break*=1mm,colback=cellbackground, colframe=cellborder]
\prompt{In}{incolor}{27}{\boxspacing}
\begin{Verbatim}[commandchars=\\\{\}]
\PY{n}{cmb\PYZus{}binned\PYZus{}spectrum}
\end{Verbatim}
\end{tcolorbox}

            \begin{tcolorbox}[breakable, size=fbox, boxrule=.5pt, pad at break*=1mm, opacityfill=0]
\prompt{Out}{outcolor}{27}{\boxspacing}
\begin{Verbatim}[commandchars=\\\{\}]
array([[4.77112240e+01, 1.47933552e+03, 5.07654876e+01, 5.07654876e+01,
        1.46111304e+03],
       [7.64716065e+01, 2.03496833e+03, 5.47101576e+01, 5.47101576e+01,
        2.06238073e+03],
       [1.05917385e+02, 2.95539416e+03, 6.49766440e+01, 6.49766440e+01,
        2.90452526e+03],
       [1.35605348e+02, 3.86951392e+03, 7.69143744e+01, 7.69143744e+01,
        3.90248963e+03],
       [1.65405597e+02, 4.88946506e+03, 8.65856259e+01, 8.65856259e+01,
        4.86135537e+03],
       [1.95266870e+02, 5.46410945e+03, 9.05533459e+01, 9.05533459e+01,
        5.53480715e+03],
       [2.25164945e+02, 5.79343954e+03, 8.71348811e+01, 8.71348811e+01,
        5.72693135e+03],
       [2.55086908e+02, 5.37288375e+03, 7.69383697e+01, 7.69383697e+01,
        5.37498113e+03],
       [2.85025248e+02, 4.62767753e+03, 6.25199141e+01, 6.25199141e+01,
        4.58832266e+03],
       [3.14975304e+02, 3.60423851e+03, 4.71589439e+01, 4.71589439e+01,
        3.59689801e+03],
       [3.44934027e+02, 2.63120029e+03, 3.38762191e+01, 3.38762191e+01,
        2.66745188e+03],
       [3.74899344e+02, 2.03305943e+03, 2.48190562e+01, 2.48190562e+01,
        2.01816864e+03],
       [4.04869791e+02, 1.75336253e+03, 2.07109083e+01, 2.07109083e+01,
        1.74834025e+03],
       [4.34844309e+02, 1.78757901e+03, 2.06713695e+01, 2.06713695e+01,
        1.82192489e+03],
       [4.64822111e+02, 2.16204649e+03, 2.27523861e+01, 2.27523861e+01,
        2.09823728e+03],
       [4.94802601e+02, 2.42208480e+03, 2.49726104e+01, 2.49726104e+01,
        2.39772998e+03],
       [5.24785320e+02, 2.57348050e+03, 2.59184118e+01, 2.59184118e+01,
        2.57144487e+03],
       [5.54769905e+02, 2.54629768e+03, 2.50510538e+01, 2.50510538e+01,
        2.55180645e+03],
       [5.84756069e+02, 2.36064530e+03, 2.27209283e+01, 2.27209283e+01,
        2.36436140e+03],
       [6.14743583e+02, 2.09543505e+03, 1.98505488e+01, 1.98505488e+01,
        2.10480607e+03],
       [6.44732258e+02, 1.88467698e+03, 1.75025264e+01, 1.75025264e+01,
        1.89096016e+03],
       [6.74721939e+02, 1.81316288e+03, 1.64306044e+01, 1.64306044e+01,
        1.81252216e+03],
       [7.04712498e+02, 1.88319393e+03, 1.67990643e+01, 1.67990643e+01,
        1.89672650e+03],
       [7.34703827e+02, 2.09713072e+03, 1.81645476e+01, 1.81645476e+01,
        2.10225612e+03],
       [7.64695836e+02, 2.31873584e+03, 1.97363538e+01, 1.97363538e+01,
        2.33880907e+03],
       [7.94688447e+02, 2.46458129e+03, 2.07062220e+01, 2.07062220e+01,
        2.50417440e+03],
       [8.24681596e+02, 2.52191260e+03, 2.05197620e+01, 2.05197620e+01,
        2.52277302e+03],
       [8.54675226e+02, 2.39412077e+03, 1.90381031e+01, 1.90381031e+01,
        2.37141912e+03],
       [8.84669287e+02, 2.08333867e+03, 1.65790348e+01, 1.65790348e+01,
        2.08335227e+03],
       [9.14663738e+02, 1.74072841e+03, 1.37374917e+01, 1.37374917e+01,
        1.73253226e+03],
       [9.44658541e+02, 1.41867342e+03, 1.11420483e+01, 1.11420483e+01,
        1.40502882e+03],
       [9.74653663e+02, 1.17295424e+03, 9.28646475e+00, 9.28646475e+00,
        1.17014623e+03],
       [1.00464908e+03, 1.06240462e+03, 8.37987000e+00, 8.37987000e+00,
        1.05887288e+03],
       [1.03464476e+03, 1.04786241e+03, 8.30074773e+00, 8.30074773e+00,
        1.05936420e+03],
       [1.06464068e+03, 1.13235191e+03, 8.68575434e+00, 8.68575434e+00,
        1.12669220e+03],
       [1.09463683e+03, 1.21101851e+03, 9.11274984e+00, 9.11274984e+00,
        1.20298244e+03],
       [1.12463318e+03, 1.23187224e+03, 9.26050628e+00, 9.26050628e+00,
        1.23858790e+03],
       [1.15462972e+03, 1.20578050e+03, 8.98528122e+00, 8.98528122e+00,
        1.20804315e+03],
       [1.18462643e+03, 1.11792662e+03, 8.33171031e+00, 8.33171031e+00,
        1.11502109e+03],
       [1.21462331e+03, 9.68422243e+02, 7.54508665e+00, 7.54508665e+00,
        9.86438951e+02],
       [1.24462034e+03, 8.64397967e+02, 6.64322644e+00, 6.64322644e+00,
        8.59469330e+02],
       [1.27461751e+03, 7.61605936e+02, 5.98585735e+00, 5.98585735e+00,
        7.66845300e+02],
       [1.30461481e+03, 7.32339633e+02, 5.68666086e+00, 5.68666086e+00,
        7.25319871e+02],
       [1.33461223e+03, 7.37290337e+02, 5.71800485e+00, 5.71800485e+00,
        7.31898515e+02],
       [1.36460976e+03, 7.74925851e+02, 5.94089388e+00, 5.94089388e+00,
        7.67047643e+02],
       [1.39460740e+03, 8.06625346e+02, 6.17243469e+00, 6.17243469e+00,
        8.03588527e+02],
       [1.42460514e+03, 8.09057064e+02, 6.25632882e+00, 6.25632882e+00,
        8.16784389e+02],
       [1.45460297e+03, 7.77751089e+02, 6.11143430e+00, 6.11143430e+00,
        7.92658706e+02],
       [1.48460089e+03, 7.28943580e+02, 5.74544667e+00, 5.74544667e+00,
        7.31230681e+02],
       [1.51459890e+03, 6.48288645e+02, 5.24258624e+00, 5.24258624e+00,
        6.44972815e+02],
       [1.54459698e+03, 5.51286111e+02, 4.72393254e+00, 4.72393254e+00,
        5.52838213e+02],
       [1.57459513e+03, 4.76600995e+02, 4.30079600e+00, 4.30079600e+00,
        4.73410287e+02],
       [1.60459335e+03, 4.19541270e+02, 4.04442416e+00, 4.04442416e+00,
        4.18888542e+02],
       [1.63459164e+03, 3.95211496e+02, 3.96833591e+00, 3.96833591e+00,
        3.92152600e+02],
       [1.66458999e+03, 3.91657847e+02, 4.03423921e+00, 4.03423921e+00,
        3.87313797e+02],
       [1.69458839e+03, 3.92947519e+02, 4.17356282e+00, 4.17356282e+00,
        3.93097715e+02],
       [1.72458686e+03, 3.97735803e+02, 4.31665510e+00, 4.31665510e+00,
        3.97391812e+02],
       [1.75458537e+03, 3.83011743e+02, 4.41586492e+00, 4.41586492e+00,
        3.91427360e+02],
       [1.78458394e+03, 3.75391143e+02, 4.45733427e+00, 4.45733427e+00,
        3.71983403e+02],
       [1.81458255e+03, 3.40139516e+02, 4.45892926e+00, 4.45892926e+00,
        3.41585903e+02],
       [1.84458121e+03, 3.07539268e+02, 4.45749277e+00, 4.45749277e+00,
        3.06532670e+02],
       [1.87457991e+03, 2.73859536e+02, 4.49350451e+00, 4.49350451e+00,
        2.74153525e+02],
       [1.90457865e+03, 2.49846901e+02, 4.59652807e+00, 4.59652807e+00,
        2.50006349e+02],
       [1.93457743e+03, 2.44246300e+02, 4.77758093e+00, 4.77758093e+00,
        2.36236579e+02],
       [1.96457625e+03, 2.31913142e+02, 5.02866666e+00, 5.02866666e+00,
        2.31349665e+02],
       [1.99457510e+03, 2.34658998e+02, 5.42839879e+00, 5.42839879e+00,
        2.31209089e+02],
       [2.02457399e+03, 2.31744942e+02, 5.86880339e+00, 5.86880339e+00,
        2.30818750e+02],
       [2.05457291e+03, 2.19660112e+02, 6.20024969e+00, 6.20024969e+00,
        2.26094532e+02],
       [2.08457186e+03, 2.14252650e+02, 6.53323083e+00, 6.53323083e+00,
        2.15102553e+02],
       [2.11457084e+03, 1.96716713e+02, 6.87393629e+00, 6.87393629e+00,
        1.98340298e+02],
       [2.14456985e+03, 1.76431323e+02, 7.23789713e+00, 7.23789713e+00,
        1.78237039e+02],
       [2.17456889e+03, 1.52227062e+02, 7.64417922e+00, 7.64417922e+00,
        1.58049842e+02],
       [2.20456795e+03, 1.38671109e+02, 8.10965997e+00, 8.10965997e+00,
        1.40733305e+02],
       [2.23456704e+03, 1.21459219e+02, 8.64541703e+00, 8.64541703e+00,
        1.28004092e+02],
       [2.26456615e+03, 1.16155153e+02, 9.25556670e+00, 9.25556670e+00,
        1.20051892e+02],
       [2.29456529e+03, 1.17768892e+02, 9.93861339e+00, 9.93861339e+00,
        1.15705770e+02],
       [2.32456444e+03, 1.16129438e+02, 1.06905048e+01, 1.06905048e+01,
        1.13100669e+02],
       [2.35456362e+03, 1.09714703e+02, 1.15085752e+01, 1.15085752e+01,
        1.10345705e+02],
       [2.38456282e+03, 1.20361413e+02, 1.23926722e+01, 1.23926722e+01,
        1.06139560e+02],
       [2.41456204e+03, 9.04515698e+01, 1.33433129e+01, 1.33433129e+01,
        1.00105444e+02],
       [2.44456128e+03, 1.03080972e+02, 1.43819739e+01, 1.43819739e+01,
        9.26666904e+01],
       [2.47456054e+03, 7.36743738e+01, 1.55113561e+01, 1.55113561e+01,
        8.48106882e+01],
       [2.49902400e+03, 6.13290585e+01, 2.09398269e+01, 2.09398269e+01,
        7.87736295e+01]])
\end{Verbatim}
\end{tcolorbox}
        
    \hypertarget{computing-the-power-spectrum}{%
\paragraph{Computing the power
spectrum}\label{computing-the-power-spectrum}}

    \begin{tcolorbox}[breakable, size=fbox, boxrule=1pt, pad at break*=1mm,colback=cellbackground, colframe=cellborder]
\prompt{In}{incolor}{28}{\boxspacing}
\begin{Verbatim}[commandchars=\\\{\}]
\PY{n}{lmax} \PY{o}{=} \PY{l+m+mi}{3000}
\end{Verbatim}
\end{tcolorbox}

    \begin{tcolorbox}[breakable, size=fbox, boxrule=1pt, pad at break*=1mm,colback=cellbackground, colframe=cellborder]
\prompt{In}{incolor}{29}{\boxspacing}
\begin{Verbatim}[commandchars=\\\{\}]
\PY{n}{test\PYZus{}cls\PYZus{}meas\PYZus{}frommap} \PY{o}{=} \PY{n}{hp}\PY{o}{.}\PY{n}{anafast}\PY{p}{(}\PY{n}{map\PYZus{}masked}\PY{p}{,} \PY{n}{lmax}\PY{o}{=}\PY{n}{lmax}\PY{p}{,} \PY{n}{use\PYZus{}pixel\PYZus{}weights}\PY{o}{=}\PY{k+kc}{True}\PY{p}{)}    \PY{c+c1}{\PYZsh{}used \PYZdq{}use\PYZus{}pixel\PYZus{}weights=True\PYZdq{} arg to have a more precise power spectrum}
\end{Verbatim}
\end{tcolorbox}

    \begin{tcolorbox}[breakable, size=fbox, boxrule=1pt, pad at break*=1mm,colback=cellbackground, colframe=cellborder]
\prompt{In}{incolor}{30}{\boxspacing}
\begin{Verbatim}[commandchars=\\\{\}]
\PY{n}{ll} \PY{o}{=} \PY{n}{np}\PY{o}{.}\PY{n}{arange}\PY{p}{(}\PY{n}{lmax}\PY{o}{+}\PY{l+m+mi}{1}\PY{p}{)}
\PY{n}{ll}
\end{Verbatim}
\end{tcolorbox}

            \begin{tcolorbox}[breakable, size=fbox, boxrule=.5pt, pad at break*=1mm, opacityfill=0]
\prompt{Out}{outcolor}{30}{\boxspacing}
\begin{Verbatim}[commandchars=\\\{\}]
array([   0,    1,    2, {\ldots}, 2998, 2999, 3000])
\end{Verbatim}
\end{tcolorbox}
        
    \begin{tcolorbox}[breakable, size=fbox, boxrule=1pt, pad at break*=1mm,colback=cellbackground, colframe=cellborder]
\prompt{In}{incolor}{31}{\boxspacing}
\begin{Verbatim}[commandchars=\\\{\}]
\PY{n}{sky\PYZus{}fraction} \PY{o}{=} \PY{n+nb}{len}\PY{p}{(}\PY{n}{map\PYZus{}masked}\PY{o}{.}\PY{n}{compressed}\PY{p}{(}\PY{p}{)}\PY{p}{)} \PY{o}{/} \PY{n+nb}{len}\PY{p}{(}\PY{n}{map\PYZus{}masked}\PY{p}{)}
\end{Verbatim}
\end{tcolorbox}

    \begin{tcolorbox}[breakable, size=fbox, boxrule=1pt, pad at break*=1mm,colback=cellbackground, colframe=cellborder]
\prompt{In}{incolor}{32}{\boxspacing}
\begin{Verbatim}[commandchars=\\\{\}]
\PY{n+nb}{print}\PY{p}{(}\PY{l+s+sa}{f}\PY{l+s+s2}{\PYZdq{}}\PY{l+s+s2}{The map covers }\PY{l+s+si}{\PYZob{}}\PY{n}{sky\PYZus{}fraction}\PY{l+s+si}{:}\PY{l+s+s2}{.1\PYZpc{}}\PY{l+s+si}{\PYZcb{}}\PY{l+s+s2}{ of the sky}\PY{l+s+s2}{\PYZdq{}}\PY{p}{)}
\end{Verbatim}
\end{tcolorbox}

    \begin{Verbatim}[commandchars=\\\{\}]
The map covers 77.9\% of the sky
    \end{Verbatim}

    \begin{tcolorbox}[breakable, size=fbox, boxrule=1pt, pad at break*=1mm,colback=cellbackground, colframe=cellborder]
\prompt{In}{incolor}{33}{\boxspacing}
\begin{Verbatim}[commandchars=\\\{\}]
\PY{n}{plt}\PY{o}{.}\PY{n}{style}\PY{o}{.}\PY{n}{use}\PY{p}{(}\PY{l+s+s2}{\PYZdq{}}\PY{l+s+s2}{seaborn\PYZhy{}poster}\PY{l+s+s2}{\PYZdq{}}\PY{p}{)}       \PY{c+c1}{\PYZsh{}read more on style used}
\end{Verbatim}
\end{tcolorbox}

    \begin{tcolorbox}[breakable, size=fbox, boxrule=1pt, pad at break*=1mm,colback=cellbackground, colframe=cellborder]
\prompt{In}{incolor}{34}{\boxspacing}
\begin{Verbatim}[commandchars=\\\{\}]
\PY{n}{k2muK} \PY{o}{=} \PY{l+m+mf}{1e6}
\end{Verbatim}
\end{tcolorbox}

    Power spectra are generally plotted as \(D_\ell\) which is defined as
\(\dfrac{\ell(\ell+1)}{2 \pi}C_\ell\), so we need to apply that factor
to the \(C_\ell\) calculated from the map.

    \begin{tcolorbox}[breakable, size=fbox, boxrule=1pt, pad at break*=1mm,colback=cellbackground, colframe=cellborder]
\prompt{In}{incolor}{35}{\boxspacing}
\begin{Verbatim}[commandchars=\\\{\}]
\PY{n}{plt}\PY{o}{.}\PY{n}{plot}\PY{p}{(}\PY{n}{cmb\PYZus{}binned\PYZus{}spectrum}\PY{p}{[}\PY{p}{:}\PY{p}{,}\PY{l+m+mi}{0}\PY{p}{]}\PY{p}{,} \PY{n}{cmb\PYZus{}binned\PYZus{}spectrum}\PY{p}{[}\PY{p}{:}\PY{p}{,}\PY{l+m+mi}{1}\PY{p}{]}\PY{p}{,} \PY{l+s+s1}{\PYZsq{}}\PY{l+s+s1}{\PYZhy{}\PYZhy{}}\PY{l+s+s1}{\PYZsq{}}\PY{p}{,} \PY{n}{alpha}\PY{o}{=}\PY{l+m+mi}{1}\PY{p}{,} \PY{n}{label}\PY{o}{=}\PY{l+s+s1}{\PYZsq{}}\PY{l+s+s1}{Planck 2018 PS release}\PY{l+s+s1}{\PYZsq{}}\PY{p}{)}
\PY{n}{plt}\PY{o}{.}\PY{n}{plot}\PY{p}{(}\PY{n}{ll}\PY{p}{,} \PY{n}{ll}\PY{o}{*}\PY{p}{(}\PY{n}{ll}\PY{o}{+}\PY{l+m+mf}{1.}\PY{p}{)}\PY{o}{*}\PY{n}{test\PYZus{}cls\PYZus{}meas\PYZus{}frommap}\PY{o}{*}\PY{n}{k2muK}\PY{o}{*}\PY{o}{*}\PY{l+m+mi}{2}\PY{o}{/}\PY{l+m+mf}{2.}\PY{o}{/}\PY{n}{np}\PY{o}{.}\PY{n}{pi} \PY{o}{/} \PY{n}{sky\PYZus{}fraction}\PY{p}{,} \PY{l+s+s1}{\PYZsq{}}\PY{l+s+s1}{\PYZhy{}\PYZhy{}}\PY{l+s+s1}{\PYZsq{}}\PY{p}{,} \PY{n}{alpha}\PY{o}{=}\PY{l+m+mf}{0.6}\PY{p}{,} \PY{n}{label}\PY{o}{=}\PY{l+s+s1}{\PYZsq{}}\PY{l+s+s1}{Planck 2018 PS from Data Map}\PY{l+s+s1}{\PYZsq{}}\PY{p}{)}
\PY{n}{plt}\PY{o}{.}\PY{n}{xlabel}\PY{p}{(}\PY{l+s+sa}{r}\PY{l+s+s1}{\PYZsq{}}\PY{l+s+s1}{\PYZdl{}}\PY{l+s+s1}{\PYZbs{}}\PY{l+s+s1}{ell\PYZdl{}}\PY{l+s+s1}{\PYZsq{}}\PY{p}{)}
\PY{n}{plt}\PY{o}{.}\PY{n}{ylabel}\PY{p}{(}\PY{l+s+sa}{r}\PY{l+s+s1}{\PYZsq{}}\PY{l+s+s1}{\PYZdl{}D\PYZus{}}\PY{l+s+s1}{\PYZbs{}}\PY{l+s+s1}{ell\PYZti{}[}\PY{l+s+s1}{\PYZbs{}}\PY{l+s+s1}{mu K\PYZca{}2]\PYZdl{}}\PY{l+s+s1}{\PYZsq{}}\PY{p}{)}
\PY{n}{plt}\PY{o}{.}\PY{n}{grid}\PY{p}{(}\PY{p}{)}
\PY{n}{plt}\PY{o}{.}\PY{n}{legend}\PY{p}{(}\PY{n}{loc}\PY{o}{=}\PY{l+s+s1}{\PYZsq{}}\PY{l+s+s1}{best}\PY{l+s+s1}{\PYZsq{}}\PY{p}{)}
\end{Verbatim}
\end{tcolorbox}

            \begin{tcolorbox}[breakable, size=fbox, boxrule=.5pt, pad at break*=1mm, opacityfill=0]
\prompt{Out}{outcolor}{35}{\boxspacing}
\begin{Verbatim}[commandchars=\\\{\}]
<matplotlib.legend.Legend at 0x10ddcb160>
\end{Verbatim}
\end{tcolorbox}
        
    \begin{center}
    \adjustimage{max size={0.9\linewidth}{0.9\paperheight}}{output_47_1.png}
    \end{center}
    { \hspace*{\fill} \\}
    
    \hypertarget{beam-correction}{%
\subsubsection{Beam Correction}\label{beam-correction}}

    Reading the documentation of the Planck release we see that the output
has a resolution of 5 arcminutes. Therefore as a first order correction
of the beam, we can divide the power spectrum by the square of the beam
window function.

    \begin{tcolorbox}[breakable, size=fbox, boxrule=1pt, pad at break*=1mm,colback=cellbackground, colframe=cellborder]
\prompt{In}{incolor}{36}{\boxspacing}
\begin{Verbatim}[commandchars=\\\{\}]
\PY{n}{w\PYZus{}ell} \PY{o}{=} \PY{n}{hp}\PY{o}{.}\PY{n}{gauss\PYZus{}beam}\PY{p}{(}\PY{p}{(}\PY{l+m+mi}{5}\PY{o}{*}\PY{n}{u}\PY{o}{.}\PY{n}{arcmin}\PY{p}{)}\PY{o}{.}\PY{n}{to\PYZus{}value}\PY{p}{(}\PY{n}{u}\PY{o}{.}\PY{n}{radian}\PY{p}{)}\PY{p}{,} \PY{n}{lmax}\PY{o}{=}\PY{n}{lmax}\PY{p}{)}
\end{Verbatim}
\end{tcolorbox}

    \begin{tcolorbox}[breakable, size=fbox, boxrule=1pt, pad at break*=1mm,colback=cellbackground, colframe=cellborder]
\prompt{In}{incolor}{37}{\boxspacing}
\begin{Verbatim}[commandchars=\\\{\}]
\PY{n}{plt}\PY{o}{.}\PY{n}{plot}\PY{p}{(}\PY{n}{cmb\PYZus{}binned\PYZus{}spectrum}\PY{p}{[}\PY{p}{:}\PY{p}{,}\PY{l+m+mi}{0}\PY{p}{]}\PY{p}{,} \PY{n}{cmb\PYZus{}binned\PYZus{}spectrum}\PY{p}{[}\PY{p}{:}\PY{p}{,}\PY{l+m+mi}{1}\PY{p}{]}\PY{p}{,} \PY{l+s+s1}{\PYZsq{}}\PY{l+s+s1}{\PYZhy{}\PYZhy{}}\PY{l+s+s1}{\PYZsq{}}\PY{p}{,} \PY{n}{alpha}\PY{o}{=}\PY{l+m+mi}{1}\PY{p}{,} \PY{n}{label}\PY{o}{=}\PY{l+s+s1}{\PYZsq{}}\PY{l+s+s1}{Planck 2018 PS release}\PY{l+s+s1}{\PYZsq{}}\PY{p}{)}
\PY{n}{plt}\PY{o}{.}\PY{n}{plot}\PY{p}{(}\PY{n}{ll}\PY{p}{,} \PY{n}{ll}\PY{o}{*}\PY{p}{(}\PY{n}{ll}\PY{o}{+}\PY{l+m+mf}{1.}\PY{p}{)}\PY{o}{*}\PY{n}{test\PYZus{}cls\PYZus{}meas\PYZus{}frommap}\PY{o}{*}\PY{n}{k2muK}\PY{o}{*}\PY{o}{*}\PY{l+m+mi}{2}\PY{o}{/}\PY{l+m+mf}{2.}\PY{o}{/}\PY{n}{np}\PY{o}{.}\PY{n}{pi} \PY{o}{/} \PY{n}{sky\PYZus{}fraction} \PY{o}{/} \PY{n}{w\PYZus{}ell}\PY{o}{*}\PY{o}{*}\PY{l+m+mi}{2}\PY{p}{,}
         \PY{n}{alpha}\PY{o}{=}\PY{l+m+mf}{0.6}\PY{p}{,} \PY{n}{label}\PY{o}{=}\PY{l+s+s1}{\PYZsq{}}\PY{l+s+s1}{Planck 2018 PS from Data Map (beam corrected)}\PY{l+s+s1}{\PYZsq{}}\PY{p}{)}
\PY{n}{plt}\PY{o}{.}\PY{n}{xlabel}\PY{p}{(}\PY{l+s+sa}{r}\PY{l+s+s1}{\PYZsq{}}\PY{l+s+s1}{\PYZdl{}}\PY{l+s+s1}{\PYZbs{}}\PY{l+s+s1}{ell\PYZdl{}}\PY{l+s+s1}{\PYZsq{}}\PY{p}{)}
\PY{n}{plt}\PY{o}{.}\PY{n}{ylabel}\PY{p}{(}\PY{l+s+sa}{r}\PY{l+s+s1}{\PYZsq{}}\PY{l+s+s1}{\PYZdl{}D\PYZus{}}\PY{l+s+s1}{\PYZbs{}}\PY{l+s+s1}{ell\PYZti{}[}\PY{l+s+s1}{\PYZbs{}}\PY{l+s+s1}{mu K\PYZca{}2]\PYZdl{}}\PY{l+s+s1}{\PYZsq{}}\PY{p}{)}
\PY{n}{plt}\PY{o}{.}\PY{n}{grid}\PY{p}{(}\PY{p}{)}
\PY{n}{plt}\PY{o}{.}\PY{n}{legend}\PY{p}{(}\PY{n}{loc}\PY{o}{=}\PY{l+s+s1}{\PYZsq{}}\PY{l+s+s1}{best}\PY{l+s+s1}{\PYZsq{}}\PY{p}{)}\PY{p}{;}
\end{Verbatim}
\end{tcolorbox}

    \begin{center}
    \adjustimage{max size={0.9\linewidth}{0.9\paperheight}}{output_51_0.png}
    \end{center}
    { \hspace*{\fill} \\}
    
    Intermediate(Getting Clean Map used for analysis)

    \begin{tcolorbox}[breakable, size=fbox, boxrule=1pt, pad at break*=1mm,colback=cellbackground, colframe=cellborder]
\prompt{In}{incolor}{38}{\boxspacing}
\begin{Verbatim}[commandchars=\\\{\}]
\PY{k+kn}{from} \PY{n+nn}{pixell} \PY{k+kn}{import} \PY{n}{enmap}\PY{p}{,} \PY{n}{utils}
\PY{k+kn}{from} \PY{n+nn}{astropy}\PY{n+nn}{.}\PY{n+nn}{io} \PY{k+kn}{import} \PY{n}{fits}
\PY{k+kn}{from} \PY{n+nn}{astropy}\PY{n+nn}{.}\PY{n+nn}{utils}\PY{n+nn}{.}\PY{n+nn}{data} \PY{k+kn}{import} \PY{n}{get\PYZus{}pkg\PYZus{}data\PYZus{}filename}
\PY{k+kn}{from} \PY{n+nn}{astropy}\PY{n+nn}{.}\PY{n+nn}{convolution} \PY{k+kn}{import} \PY{n}{Gaussian2DKernel}
\PY{k+kn}{from} \PY{n+nn}{scipy}\PY{n+nn}{.}\PY{n+nn}{signal} \PY{k+kn}{import} \PY{n}{convolve} \PY{k}{as} \PY{n}{scipy\PYZus{}convolve}
\PY{k+kn}{from} \PY{n+nn}{astropy}\PY{n+nn}{.}\PY{n+nn}{convolution} \PY{k+kn}{import} \PY{n}{convolve}
\end{Verbatim}
\end{tcolorbox}

    \begin{tcolorbox}[breakable, size=fbox, boxrule=1pt, pad at break*=1mm,colback=cellbackground, colframe=cellborder]
\prompt{In}{incolor}{ }{\boxspacing}
\begin{Verbatim}[commandchars=\\\{\}]

\end{Verbatim}
\end{tcolorbox}

    \begin{tcolorbox}[breakable, size=fbox, boxrule=1pt, pad at break*=1mm,colback=cellbackground, colframe=cellborder]
\prompt{In}{incolor}{ }{\boxspacing}
\begin{Verbatim}[commandchars=\\\{\}]

\end{Verbatim}
\end{tcolorbox}

    \begin{tcolorbox}[breakable, size=fbox, boxrule=1pt, pad at break*=1mm,colback=cellbackground, colframe=cellborder]
\prompt{In}{incolor}{39}{\boxspacing}
\begin{Verbatim}[commandchars=\\\{\}]
\PY{c+c1}{\PYZsh{} Load the data }
\PY{n}{filename} \PY{o}{=} \PY{n}{get\PYZus{}pkg\PYZus{}data\PYZus{}filename}\PY{p}{(}\PY{l+s+s1}{\PYZsq{}}\PY{l+s+s1}{ffp10\PYZus{}newdust\PYZus{}total\PYZus{}030\PYZus{}full\PYZus{}map.fits}\PY{l+s+s1}{\PYZsq{}}\PY{p}{)}
\PY{n}{hdu} \PY{o}{=} \PY{n}{fits}\PY{o}{.}\PY{n}{open}\PY{p}{(}\PY{n}{filename}\PY{p}{)}\PY{p}{[}\PY{l+m+mi}{0}\PY{p}{]}
\end{Verbatim}
\end{tcolorbox}

    \begin{tcolorbox}[breakable, size=fbox, boxrule=1pt, pad at break*=1mm,colback=cellbackground, colframe=cellborder]
\prompt{In}{incolor}{40}{\boxspacing}
\begin{Verbatim}[commandchars=\\\{\}]
\PY{n}{hdu}
\end{Verbatim}
\end{tcolorbox}

            \begin{tcolorbox}[breakable, size=fbox, boxrule=.5pt, pad at break*=1mm, opacityfill=0]
\prompt{Out}{outcolor}{40}{\boxspacing}
\begin{Verbatim}[commandchars=\\\{\}]
<astropy.io.fits.hdu.image.PrimaryHDU at 0x142386670>
\end{Verbatim}
\end{tcolorbox}
        
    \begin{tcolorbox}[breakable, size=fbox, boxrule=1pt, pad at break*=1mm,colback=cellbackground, colframe=cellborder]
\prompt{In}{incolor}{41}{\boxspacing}
\begin{Verbatim}[commandchars=\\\{\}]
\PY{k+kn}{from} \PY{n+nn}{astropy}\PY{n+nn}{.}\PY{n+nn}{convolution} \PY{k+kn}{import} \PY{n}{Gaussian1DKernel}
\end{Verbatim}
\end{tcolorbox}

    \begin{tcolorbox}[breakable, size=fbox, boxrule=1pt, pad at break*=1mm,colback=cellbackground, colframe=cellborder]
\prompt{In}{incolor}{42}{\boxspacing}
\begin{Verbatim}[commandchars=\\\{\}]
\PY{c+c1}{\PYZsh{}the convolution module also includes built\PYZhy{}in kernels that can be imported as}

\PY{n}{gauss} \PY{o}{=} \PY{n}{Gaussian1DKernel}\PY{p}{(}\PY{n}{stddev}\PY{o}{=}\PY{l+m+mi}{2}\PY{p}{)}
\end{Verbatim}
\end{tcolorbox}

    \begin{tcolorbox}[breakable, size=fbox, boxrule=1pt, pad at break*=1mm,colback=cellbackground, colframe=cellborder]
\prompt{In}{incolor}{43}{\boxspacing}
\begin{Verbatim}[commandchars=\\\{\}]
\PY{n}{gauss}\PY{o}{.}\PY{n}{array} 
\end{Verbatim}
\end{tcolorbox}

            \begin{tcolorbox}[breakable, size=fbox, boxrule=.5pt, pad at break*=1mm, opacityfill=0]
\prompt{Out}{outcolor}{43}{\boxspacing}
\begin{Verbatim}[commandchars=\\\{\}]
array([6.69151129e-05, 4.36341348e-04, 2.21592421e-03, 8.76415025e-03,
       2.69954833e-02, 6.47587978e-02, 1.20985362e-01, 1.76032663e-01,
       1.99471140e-01, 1.76032663e-01, 1.20985362e-01, 6.47587978e-02,
       2.69954833e-02, 8.76415025e-03, 2.21592421e-03, 4.36341348e-04,
       6.69151129e-05])
\end{Verbatim}
\end{tcolorbox}
        
    The kernel can then be used directly when calling convolve()

    \begin{tcolorbox}[breakable, size=fbox, boxrule=1pt, pad at break*=1mm,colback=cellbackground, colframe=cellborder]
\prompt{In}{incolor}{ }{\boxspacing}
\begin{Verbatim}[commandchars=\\\{\}]

\end{Verbatim}
\end{tcolorbox}

    \begin{tcolorbox}[breakable, size=fbox, boxrule=1pt, pad at break*=1mm,colback=cellbackground, colframe=cellborder]
\prompt{In}{incolor}{ }{\boxspacing}
\begin{Verbatim}[commandchars=\\\{\}]

\end{Verbatim}
\end{tcolorbox}

    \hypertarget{bingo}{%
\subsubsection{Bingo}\label{bingo}}

    \begin{tcolorbox}[breakable, size=fbox, boxrule=1pt, pad at break*=1mm,colback=cellbackground, colframe=cellborder]
\prompt{In}{incolor}{44}{\boxspacing}
\begin{Verbatim}[commandchars=\\\{\}]
\PY{n}{NSIDE} \PY{o}{=} \PY{l+m+mi}{1024}
\PY{n+nb}{print}\PY{p}{(}\PY{l+s+s2}{\PYZdq{}}\PY{l+s+s2}{Approximate resolution at NSIDE }\PY{l+s+si}{\PYZob{}\PYZcb{}}\PY{l+s+s2}{ is }\PY{l+s+si}{\PYZob{}:.2\PYZcb{}}\PY{l+s+s2}{ deg}\PY{l+s+s2}{\PYZdq{}}\PY{o}{.}\PY{n}{format}\PY{p}{(}
        \PY{n}{NSIDE}\PY{p}{,} \PY{n}{hp}\PY{o}{.}\PY{n}{nside2resol}\PY{p}{(}\PY{n}{NSIDE}\PY{p}{,} \PY{n}{arcmin}\PY{o}{=}\PY{k+kc}{True}\PY{p}{)} \PY{o}{/} \PY{l+m+mi}{60}
\PY{p}{)} \PY{p}{)}
\end{Verbatim}
\end{tcolorbox}

    \begin{Verbatim}[commandchars=\\\{\}]
Approximate resolution at NSIDE 1024 is 0.057 deg
    \end{Verbatim}

    \begin{tcolorbox}[breakable, size=fbox, boxrule=1pt, pad at break*=1mm,colback=cellbackground, colframe=cellborder]
\prompt{In}{incolor}{45}{\boxspacing}
\begin{Verbatim}[commandchars=\\\{\}]
\PY{n}{NPIX} \PY{o}{=} \PY{n}{hp}\PY{o}{.}\PY{n}{nside2npix}\PY{p}{(}\PY{n}{NSIDE}\PY{p}{)} 
\PY{n+nb}{print}\PY{p}{(}\PY{n}{NPIX}\PY{p}{)}
\end{Verbatim}
\end{tcolorbox}

    \begin{Verbatim}[commandchars=\\\{\}]
12582912
    \end{Verbatim}

    getting alms of map(sample) used from Planck

    \begin{tcolorbox}[breakable, size=fbox, boxrule=1pt, pad at break*=1mm,colback=cellbackground, colframe=cellborder]
\prompt{In}{incolor}{46}{\boxspacing}
\begin{Verbatim}[commandchars=\\\{\}]
\PY{n}{hp}\PY{o}{.}\PY{n}{sphtfunc}\PY{o}{.}\PY{n}{map2alm}\PY{p}{(}\PY{n}{map\PYZus{}sample}\PY{p}{,} \PY{n}{lmax}\PY{o}{=}\PY{l+m+mi}{1300}\PY{p}{,} \PY{n}{mmax}\PY{o}{=}\PY{k+kc}{None}\PY{p}{,} \PY{n+nb}{iter}\PY{o}{=}\PY{l+m+mi}{3}\PY{p}{,} \PY{n}{pol}\PY{o}{=}\PY{k+kc}{True}\PY{p}{,} \PY{n}{use\PYZus{}weights}\PY{o}{=}\PY{k+kc}{False}\PY{p}{,} \PY{n}{datapath}\PY{o}{=}\PY{k+kc}{None}\PY{p}{,} \PY{n}{gal\PYZus{}cut}\PY{o}{=}\PY{l+m+mi}{0}\PY{p}{,} \PY{n}{use\PYZus{}pixel\PYZus{}weights}\PY{o}{=}\PY{k+kc}{False}\PY{p}{)}
\end{Verbatim}
\end{tcolorbox}

            \begin{tcolorbox}[breakable, size=fbox, boxrule=.5pt, pad at break*=1mm, opacityfill=0]
\prompt{Out}{outcolor}{46}{\boxspacing}
\begin{Verbatim}[commandchars=\\\{\}]
array([ 1.16211765e-03+0.00000000e+00j, -1.29597927e-05+0.00000000e+00j,
       -1.32805698e-03+0.00000000e+00j, {\ldots},
        2.49503412e-08-4.73206538e-08j, -1.51505776e-08-3.88394977e-08j,
       -2.92800850e-08-3.33140654e-08j])
\end{Verbatim}
\end{tcolorbox}
        
    \hypertarget{nb-the-pixels-which-have-the-special-unseen-value-are-replaced-by-zeros-before-spherical-harmonic-transform.-they-are-converted-back-to-unseen-value-so-that-the-input-maps-are-not-modified.-each-map-have-its-own-independent-mask.}{%
\subsubsection{NB: The pixels which have the special UNSEEN value are
replaced by zeros before spherical harmonic transform. They are
converted back to UNSEEN value, so that the input maps are not modified.
Each map have its own, independent
mask.}\label{nb-the-pixels-which-have-the-special-unseen-value-are-replaced-by-zeros-before-spherical-harmonic-transform.-they-are-converted-back-to-unseen-value-so-that-the-input-maps-are-not-modified.-each-map-have-its-own-independent-mask.}}

    \begin{tcolorbox}[breakable, size=fbox, boxrule=1pt, pad at break*=1mm,colback=cellbackground, colframe=cellborder]
\prompt{In}{incolor}{101}{\boxspacing}
\begin{Verbatim}[commandchars=\\\{\}]
\PY{c+c1}{\PYZsh{}pol=false}
\PY{n}{alm} \PY{o}{=} \PY{n}{hp}\PY{o}{.}\PY{n}{sphtfunc}\PY{o}{.}\PY{n}{map2alm}\PY{p}{(}\PY{n}{map\PYZus{}sample}\PY{p}{,} \PY{n}{lmax}\PY{o}{=}\PY{l+m+mi}{1300}\PY{p}{,} \PY{n}{mmax}\PY{o}{=}\PY{k+kc}{None}\PY{p}{,} \PY{n+nb}{iter}\PY{o}{=}\PY{l+m+mi}{3}\PY{p}{,} \PY{n}{pol}\PY{o}{=}\PY{k+kc}{False}\PY{p}{,} \PY{n}{use\PYZus{}weights}\PY{o}{=}\PY{k+kc}{False}\PY{p}{,} \PY{n}{datapath}\PY{o}{=}\PY{k+kc}{None}\PY{p}{,} \PY{n}{gal\PYZus{}cut}\PY{o}{=}\PY{l+m+mi}{0}\PY{p}{,} \PY{n}{use\PYZus{}pixel\PYZus{}weights}\PY{o}{=}\PY{k+kc}{False}\PY{p}{)}
\end{Verbatim}
\end{tcolorbox}

    \begin{tcolorbox}[breakable, size=fbox, boxrule=1pt, pad at break*=1mm,colback=cellbackground, colframe=cellborder]
\prompt{In}{incolor}{102}{\boxspacing}
\begin{Verbatim}[commandchars=\\\{\}]
\PY{n}{alm}
\end{Verbatim}
\end{tcolorbox}

            \begin{tcolorbox}[breakable, size=fbox, boxrule=.5pt, pad at break*=1mm, opacityfill=0]
\prompt{Out}{outcolor}{102}{\boxspacing}
\begin{Verbatim}[commandchars=\\\{\}]
array([ 1.16211765e-03+0.00000000e+00j, -1.29597927e-05+0.00000000e+00j,
       -1.32805698e-03+0.00000000e+00j, {\ldots},
        2.49503412e-08-4.73206538e-08j, -1.51505776e-08-3.88394977e-08j,
       -2.92800850e-08-3.33140654e-08j])
\end{Verbatim}
\end{tcolorbox}
        
    " Multiply alm by a function of l. The function is assumed to be zero
where not defined.

Parameters almarray The alm to multiply

flarray The function (at l=0..fl.size-1) by which alm must be
multiplied.

mmaxNone or int, optional The maximum m defining the alm layout.
Default: lmax.

inplacebool, optional If True, modify the given alm, otherwise make a
copy before multiplying.

Returns almarray The modified alm, either a new array or a reference to
input alm, if inplace is True. "

    \begin{tcolorbox}[breakable, size=fbox, boxrule=1pt, pad at break*=1mm,colback=cellbackground, colframe=cellborder]
\prompt{In}{incolor}{103}{\boxspacing}
\begin{Verbatim}[commandchars=\\\{\}]
\PY{n}{np}\PY{o}{.}\PY{n}{shape}\PY{p}{(}\PY{n}{alm}\PY{p}{)}
\end{Verbatim}
\end{tcolorbox}

            \begin{tcolorbox}[breakable, size=fbox, boxrule=.5pt, pad at break*=1mm, opacityfill=0]
\prompt{Out}{outcolor}{103}{\boxspacing}
\begin{Verbatim}[commandchars=\\\{\}]
(846951,)
\end{Verbatim}
\end{tcolorbox}
        
    \begin{tcolorbox}[breakable, size=fbox, boxrule=1pt, pad at break*=1mm,colback=cellbackground, colframe=cellborder]
\prompt{In}{incolor}{ }{\boxspacing}
\begin{Verbatim}[commandchars=\\\{\}]

\end{Verbatim}
\end{tcolorbox}

    \begin{tcolorbox}[breakable, size=fbox, boxrule=1pt, pad at break*=1mm,colback=cellbackground, colframe=cellborder]
\prompt{In}{incolor}{104}{\boxspacing}
\begin{Verbatim}[commandchars=\\\{\}]
\PY{n}{alm}
\end{Verbatim}
\end{tcolorbox}

            \begin{tcolorbox}[breakable, size=fbox, boxrule=.5pt, pad at break*=1mm, opacityfill=0]
\prompt{Out}{outcolor}{104}{\boxspacing}
\begin{Verbatim}[commandchars=\\\{\}]
array([ 1.16211765e-03+0.00000000e+00j, -1.29597927e-05+0.00000000e+00j,
       -1.32805698e-03+0.00000000e+00j, {\ldots},
        2.49503412e-08-4.73206538e-08j, -1.51505776e-08-3.88394977e-08j,
       -2.92800850e-08-3.33140654e-08j])
\end{Verbatim}
\end{tcolorbox}
        
    \begin{tcolorbox}[breakable, size=fbox, boxrule=1pt, pad at break*=1mm,colback=cellbackground, colframe=cellborder]
\prompt{In}{incolor}{105}{\boxspacing}
\begin{Verbatim}[commandchars=\\\{\}]
\PY{c+c1}{\PYZsh{}fl = w\PYZus{}ell   \PYZsh{}see power spectra step, deriv of beam function used as fl (function of l)}
\end{Verbatim}
\end{tcolorbox}

    \begin{tcolorbox}[breakable, size=fbox, boxrule=1pt, pad at break*=1mm,colback=cellbackground, colframe=cellborder]
\prompt{In}{incolor}{106}{\boxspacing}
\begin{Verbatim}[commandchars=\\\{\}]
\PY{c+c1}{\PYZsh{}hp.sphtfunc.almxfl(alm , fl, mmax=None, inplace=True)}
\end{Verbatim}
\end{tcolorbox}

    \begin{tcolorbox}[breakable, size=fbox, boxrule=1pt, pad at break*=1mm,colback=cellbackground, colframe=cellborder]
\prompt{In}{incolor}{107}{\boxspacing}
\begin{Verbatim}[commandchars=\\\{\}]
\PY{n}{alm}
\end{Verbatim}
\end{tcolorbox}

            \begin{tcolorbox}[breakable, size=fbox, boxrule=.5pt, pad at break*=1mm, opacityfill=0]
\prompt{Out}{outcolor}{107}{\boxspacing}
\begin{Verbatim}[commandchars=\\\{\}]
array([ 1.16211765e-03+0.00000000e+00j, -1.29597927e-05+0.00000000e+00j,
       -1.32805698e-03+0.00000000e+00j, {\ldots},
        2.49503412e-08-4.73206538e-08j, -1.51505776e-08-3.88394977e-08j,
       -2.92800850e-08-3.33140654e-08j])
\end{Verbatim}
\end{tcolorbox}
        
    applying a spherical harmonic transform back to map space (hp.alm2map),
using Nside = 1024.

    \begin{tcolorbox}[breakable, size=fbox, boxrule=1pt, pad at break*=1mm,colback=cellbackground, colframe=cellborder]
\prompt{In}{incolor}{108}{\boxspacing}
\begin{Verbatim}[commandchars=\\\{\}]
\PY{c+c1}{\PYZsh{}hp.sphtfunc.alm2map(alm, nside=1024, lmax=lmax, mmax=None, pixwin=False, fwhm=0.0, sigma=None, pol=True, inplace=False, verbose=True)}
\end{Verbatim}
\end{tcolorbox}

    \begin{tcolorbox}[breakable, size=fbox, boxrule=1pt, pad at break*=1mm,colback=cellbackground, colframe=cellborder]
\prompt{In}{incolor}{91}{\boxspacing}
\begin{Verbatim}[commandchars=\\\{\}]
\PY{c+c1}{\PYZsh{}mmax = 1300}
\PY{c+c1}{\PYZsh{}checking size of alm}
\PY{p}{(}\PY{n}{mmax} \PY{o}{*} \PY{p}{(}\PY{l+m+mi}{2} \PY{o}{*} \PY{n}{lmax} \PY{o}{+} \PY{l+m+mi}{1} \PY{o}{\PYZhy{}} \PY{n}{mmax}\PY{p}{)}\PY{p}{)} \PY{o}{/} \PY{p}{(}\PY{l+m+mi}{2} \PY{o}{+} \PY{n}{lmax} \PY{o}{+} \PY{l+m+mi}{1}\PY{p}{)}
\end{Verbatim}
\end{tcolorbox}

            \begin{tcolorbox}[breakable, size=fbox, boxrule=.5pt, pad at break*=1mm, opacityfill=0]
\prompt{Out}{outcolor}{91}{\boxspacing}
\begin{Verbatim}[commandchars=\\\{\}]
2035.064935064935
\end{Verbatim}
\end{tcolorbox}
        
    \begin{tcolorbox}[breakable, size=fbox, boxrule=1pt, pad at break*=1mm,colback=cellbackground, colframe=cellborder]
\prompt{In}{incolor}{94}{\boxspacing}
\begin{Verbatim}[commandchars=\\\{\}]
\PY{c+c1}{\PYZsh{}resize alm}
\PY{c+c1}{\PYZsh{}alm\PYZus{}resize= np.resize(alm,2035)}
\PY{c+c1}{\PYZsh{}alm\PYZus{}resize}
\end{Verbatim}
\end{tcolorbox}

    \begin{tcolorbox}[breakable, size=fbox, boxrule=1pt, pad at break*=1mm,colback=cellbackground, colframe=cellborder]
\prompt{In}{incolor}{115}{\boxspacing}
\begin{Verbatim}[commandchars=\\\{\}]
\PY{c+c1}{\PYZsh{}pol=false polarisatn}
\PY{c+c1}{\PYZsh{}hp.sphtfunc.alm2map(alm,nside=16, lmax=lmax, mmax=None, pixwin=False, fwhm=0.0, sigma=None, pol=False, inplace=False, verbose=True)}
\PY{n}{new\PYZus{}map}\PY{o}{=}\PY{n}{hp}\PY{o}{.}\PY{n}{sphtfunc}\PY{o}{.}\PY{n}{alm2map}\PY{p}{(}\PY{n}{alm}\PY{p}{,}\PY{n}{nside}\PY{o}{=}\PY{l+m+mi}{1024}\PY{p}{)} \PY{c+c1}{\PYZsh{}, mmax=None, pixwin=False, fwhm=0.0, sigma=None, pol=False, inplace=False, verbose=True)}
\PY{n}{new\PYZus{}map}
\end{Verbatim}
\end{tcolorbox}

            \begin{tcolorbox}[breakable, size=fbox, boxrule=.5pt, pad at break*=1mm, opacityfill=0]
\prompt{Out}{outcolor}{115}{\boxspacing}
\begin{Verbatim}[commandchars=\\\{\}]
array([-0.00018566, -0.00016411, -0.00019511, {\ldots}, -0.00013042,
       -0.0001177 , -0.0001039 ])
\end{Verbatim}
\end{tcolorbox}
        
    \begin{tcolorbox}[breakable, size=fbox, boxrule=1pt, pad at break*=1mm,colback=cellbackground, colframe=cellborder]
\prompt{In}{incolor}{116}{\boxspacing}
\begin{Verbatim}[commandchars=\\\{\}]
\PY{n}{np}\PY{o}{.}\PY{n}{shape}\PY{p}{(}\PY{n}{new\PYZus{}map}\PY{p}{)}
\end{Verbatim}
\end{tcolorbox}

            \begin{tcolorbox}[breakable, size=fbox, boxrule=.5pt, pad at break*=1mm, opacityfill=0]
\prompt{Out}{outcolor}{116}{\boxspacing}
\begin{Verbatim}[commandchars=\\\{\}]
(12582912,)
\end{Verbatim}
\end{tcolorbox}
        
    \begin{tcolorbox}[breakable, size=fbox, boxrule=1pt, pad at break*=1mm,colback=cellbackground, colframe=cellborder]
\prompt{In}{incolor}{117}{\boxspacing}
\begin{Verbatim}[commandchars=\\\{\}]
\PY{n}{np}\PY{o}{.}\PY{n}{shape}\PY{p}{(}\PY{n}{alm}\PY{p}{)}
\end{Verbatim}
\end{tcolorbox}

            \begin{tcolorbox}[breakable, size=fbox, boxrule=.5pt, pad at break*=1mm, opacityfill=0]
\prompt{Out}{outcolor}{117}{\boxspacing}
\begin{Verbatim}[commandchars=\\\{\}]
(846951,)
\end{Verbatim}
\end{tcolorbox}
        
    \begin{tcolorbox}[breakable, size=fbox, boxrule=1pt, pad at break*=1mm,colback=cellbackground, colframe=cellborder]
\prompt{In}{incolor}{118}{\boxspacing}
\begin{Verbatim}[commandchars=\\\{\}]
\PY{n}{np}\PY{o}{.}\PY{n}{shape}\PY{p}{(}\PY{n}{map\PYZus{}sample}\PY{p}{)}
\end{Verbatim}
\end{tcolorbox}

            \begin{tcolorbox}[breakable, size=fbox, boxrule=.5pt, pad at break*=1mm, opacityfill=0]
\prompt{Out}{outcolor}{118}{\boxspacing}
\begin{Verbatim}[commandchars=\\\{\}]
(12582912,)
\end{Verbatim}
\end{tcolorbox}
        
    \hypertarget{visualise}{%
\subsection{visualise}\label{visualise}}

    \begin{tcolorbox}[breakable, size=fbox, boxrule=1pt, pad at break*=1mm,colback=cellbackground, colframe=cellborder]
\prompt{In}{incolor}{ }{\boxspacing}
\begin{Verbatim}[commandchars=\\\{\}]

\end{Verbatim}
\end{tcolorbox}

    Alternative approach for alm2map \#\#\#\# Rem to divide by hp.pixwin
(Unresolved)

    \begin{tcolorbox}[breakable, size=fbox, boxrule=1pt, pad at break*=1mm,colback=cellbackground, colframe=cellborder]
\prompt{In}{incolor}{61}{\boxspacing}
\begin{Verbatim}[commandchars=\\\{\}]
\PY{n}{hp}\PY{o}{.}\PY{n}{sphtfunc}\PY{o}{.}\PY{n}{alm2cl}\PY{p}{(}\PY{n}{alm}\PY{p}{,} \PY{n}{alms2}\PY{o}{=}\PY{k+kc}{None}\PY{p}{,} \PY{n}{lmax}\PY{o}{=}\PY{k+kc}{None}\PY{p}{,} \PY{n}{mmax}\PY{o}{=}\PY{k+kc}{None}\PY{p}{,} \PY{n}{lmax\PYZus{}out}\PY{o}{=}\PY{k+kc}{None}\PY{p}{,} \PY{n}{nspec}\PY{o}{=}\PY{k+kc}{None}\PY{p}{)}
\PY{c+c1}{\PYZsh{}Computes (cross\PYZhy{})spectra from alm(s). If alm2 is given, cross\PYZhy{}spectra between alm and alm2 are computed. If alm (and alm2 if provided) contains n alm, then n(n+1)/2 auto and cross\PYZhy{}spectra are returned}
\end{Verbatim}
\end{tcolorbox}

            \begin{tcolorbox}[breakable, size=fbox, boxrule=.5pt, pad at break*=1mm, opacityfill=0]
\prompt{Out}{outcolor}{61}{\boxspacing}
\begin{Verbatim}[commandchars=\\\{\}]
array([1.35051743e-06, 2.53453754e-07, 3.84581273e-07, {\ldots},
       3.14515626e-15, 3.22686188e-15, 3.20514566e-15])
\end{Verbatim}
\end{tcolorbox}
        
    \begin{tcolorbox}[breakable, size=fbox, boxrule=1pt, pad at break*=1mm,colback=cellbackground, colframe=cellborder]
\prompt{In}{incolor}{62}{\boxspacing}
\begin{Verbatim}[commandchars=\\\{\}]
\PY{n}{cl} \PY{o}{=} \PY{n}{hp}\PY{o}{.}\PY{n}{sphtfunc}\PY{o}{.}\PY{n}{alm2cl}\PY{p}{(}\PY{n}{alm}\PY{p}{,} \PY{n}{alms2}\PY{o}{=}\PY{k+kc}{None}\PY{p}{,} \PY{n}{lmax}\PY{o}{=}\PY{k+kc}{None}\PY{p}{,} \PY{n}{mmax}\PY{o}{=}\PY{k+kc}{None}\PY{p}{,} \PY{n}{lmax\PYZus{}out}\PY{o}{=}\PY{k+kc}{None}\PY{p}{,} \PY{n}{nspec}\PY{o}{=}\PY{k+kc}{None}\PY{p}{)}
\PY{n}{cl}
\end{Verbatim}
\end{tcolorbox}

            \begin{tcolorbox}[breakable, size=fbox, boxrule=.5pt, pad at break*=1mm, opacityfill=0]
\prompt{Out}{outcolor}{62}{\boxspacing}
\begin{Verbatim}[commandchars=\\\{\}]
array([1.35051743e-06, 2.53453754e-07, 3.84581273e-07, {\ldots},
       3.14515626e-15, 3.22686188e-15, 3.20514566e-15])
\end{Verbatim}
\end{tcolorbox}
        
    \begin{tcolorbox}[breakable, size=fbox, boxrule=1pt, pad at break*=1mm,colback=cellbackground, colframe=cellborder]
\prompt{In}{incolor}{65}{\boxspacing}
\begin{Verbatim}[commandchars=\\\{\}]
\PY{n}{alm2map} \PY{o}{=} \PY{n}{hp}\PY{o}{.}\PY{n}{sphtfunc}\PY{o}{.}\PY{n}{synfast}\PY{p}{(}\PY{n}{cl}\PY{p}{,} \PY{n}{nside}\PY{o}{=}\PY{l+m+mi}{1024}\PY{p}{,} \PY{n}{lmax}\PY{o}{=}\PY{l+m+mi}{1300}\PY{p}{,} \PY{n}{mmax}\PY{o}{=}\PY{k+kc}{None}\PY{p}{,} \PY{n}{alm}\PY{o}{=}\PY{k+kc}{False}\PY{p}{,} \PY{n}{pol}\PY{o}{=}\PY{k+kc}{True}\PY{p}{,} \PY{n}{pixwin}\PY{o}{=}\PY{k+kc}{False}\PY{p}{,} \PY{n}{fwhm}\PY{o}{=}\PY{l+m+mf}{0.0}\PY{p}{,} \PY{n}{sigma}\PY{o}{=}\PY{k+kc}{None}\PY{p}{,} \PY{n}{new}\PY{o}{=}\PY{k+kc}{False}\PY{p}{,} \PY{n}{verbose}\PY{o}{=}\PY{k+kc}{True}\PY{p}{)}
\PY{n}{alm2map}
\end{Verbatim}
\end{tcolorbox}

            \begin{tcolorbox}[breakable, size=fbox, boxrule=.5pt, pad at break*=1mm, opacityfill=0]
\prompt{Out}{outcolor}{65}{\boxspacing}
\begin{Verbatim}[commandchars=\\\{\}]
array([0.00039491, 0.00053628, 0.00071575, {\ldots}, 0.0001484 , 0.00048541,
       0.00060199])
\end{Verbatim}
\end{tcolorbox}
        
    Creates a map(s) from cl(s). Parameters cls {[}array or tuple of
array{]} A cl or a list of cl (either 4 or 6, see synalm()) nside
{[}int, scalar{]} The nside of the output map(s) lmax {[}int, scalar,
optional{]} Maximum l for alm. Default: min of 3*nside-1 or length of
the cls - 1 mmax {[}int, scalar, optional{]} Maximum m for alm. alm
{[}bool, scalar, optional{]} If True, return also alm(s). Default:
False. pol {[}bool, optional{]} If True, assumes input cls are TEB and
correlation. Output will be TQU maps. (input must be 1, 4 or 6 cl's) If
False, fields are assumed to be described by spin 0 spherical harmonics.
(input can be any number of cl's) If there is only one input cl, it has
no effect. Default: True. pixwin {[}bool, scalar, optional{]} If True,
convolve the alm by the pixel window function. Default: False. fwhm
{[}float, scalar, optional{]} The fwhm of the Gaussian used to smooth
the map (applied on alm) {[}in radians{]} sigma {[}float, scalar,
optional{]} The sigma of the Gaussian used to smooth the map (applied on
alm) {[}in radians{]} Returns maps {[}array or tuple of arrays{]} The
output map (possibly list of maps if polarized input). or, if alm is
True, a tuple of (map,alm) (alm possibly a list of alm if polarized
input)

    \begin{tcolorbox}[breakable, size=fbox, boxrule=1pt, pad at break*=1mm,colback=cellbackground, colframe=cellborder]
\prompt{In}{incolor}{68}{\boxspacing}
\begin{Verbatim}[commandchars=\\\{\}]
\PY{c+c1}{\PYZsh{}estimating covariance matrix}
\PY{c+c1}{\PYZsh{}np.cov(alm2map, y=None, rowvar=True, bias=False, ddof=None, fweights=None, aweights=None, dtype=None) NOT helpful}
\end{Verbatim}
\end{tcolorbox}

    \begin{tcolorbox}[breakable, size=fbox, boxrule=1pt, pad at break*=1mm,colback=cellbackground, colframe=cellborder]
\prompt{In}{incolor}{70}{\boxspacing}
\begin{Verbatim}[commandchars=\\\{\}]
\PY{n}{alm2map}
\end{Verbatim}
\end{tcolorbox}

            \begin{tcolorbox}[breakable, size=fbox, boxrule=.5pt, pad at break*=1mm, opacityfill=0]
\prompt{Out}{outcolor}{70}{\boxspacing}
\begin{Verbatim}[commandchars=\\\{\}]
array([0.00039491, 0.00053628, 0.00071575, {\ldots}, 0.0001484 , 0.00048541,
       0.00060199])
\end{Verbatim}
\end{tcolorbox}
        
    \begin{tcolorbox}[breakable, size=fbox, boxrule=1pt, pad at break*=1mm,colback=cellbackground, colframe=cellborder]
\prompt{In}{incolor}{120}{\boxspacing}
\begin{Verbatim}[commandchars=\\\{\}]
\PY{n}{hp}\PY{o}{.}\PY{n}{mollview}\PY{p}{(}
    \PY{n}{new\PYZus{}map}\PY{p}{,}
    \PY{n}{coord}\PY{o}{=}\PY{p}{[}\PY{l+s+s2}{\PYZdq{}}\PY{l+s+s2}{G}\PY{l+s+s2}{\PYZdq{}}\PY{p}{]}\PY{p}{,} \PY{c+c1}{\PYZsh{}galactic only}
    \PY{n}{title}\PY{o}{=}\PY{l+s+s2}{\PYZdq{}}\PY{l+s+s2}{Visualised sample}\PY{l+s+s2}{\PYZdq{}}\PY{p}{,}
    \PY{n}{unit}\PY{o}{=}\PY{l+s+s2}{\PYZdq{}}\PY{l+s+s2}{mK}\PY{l+s+s2}{\PYZdq{}}\PY{p}{,}
    \PY{n}{norm}\PY{o}{=}\PY{l+s+s2}{\PYZdq{}}\PY{l+s+s2}{hist}\PY{l+s+s2}{\PYZdq{}}\PY{p}{,}
    \PY{n+nb}{min}\PY{o}{=}\PY{o}{\PYZhy{}}\PY{l+m+mf}{0.01}\PY{p}{,}
    \PY{n+nb}{max}\PY{o}{=}\PY{l+m+mf}{0.01}\PY{p}{,}
\PY{p}{)}
\PY{n}{hp}\PY{o}{.}\PY{n}{graticule}\PY{p}{(}\PY{p}{)}
\end{Verbatim}
\end{tcolorbox}

    \begin{center}
    \adjustimage{max size={0.9\linewidth}{0.9\paperheight}}{output_96_0.png}
    \end{center}
    { \hspace*{\fill} \\}
    
    \begin{tcolorbox}[breakable, size=fbox, boxrule=1pt, pad at break*=1mm,colback=cellbackground, colframe=cellborder]
\prompt{In}{incolor}{122}{\boxspacing}
\begin{Verbatim}[commandchars=\\\{\}]
\PY{n}{hp}\PY{o}{.}\PY{n}{mollview}\PY{p}{(}
    \PY{n}{map\PYZus{}sample}\PY{p}{,}
    \PY{n}{coord}\PY{o}{=}\PY{p}{[}\PY{l+s+s2}{\PYZdq{}}\PY{l+s+s2}{G}\PY{l+s+s2}{\PYZdq{}}\PY{p}{]}\PY{p}{,}
    \PY{n}{title}\PY{o}{=}\PY{l+s+s2}{\PYZdq{}}\PY{l+s+s2}{Histogram equalized Galactic Sample 30Hz}\PY{l+s+s2}{\PYZdq{}}\PY{p}{,}
    \PY{n}{unit}\PY{o}{=}\PY{l+s+s2}{\PYZdq{}}\PY{l+s+s2}{mK}\PY{l+s+s2}{\PYZdq{}}\PY{p}{,}
    \PY{n}{norm}\PY{o}{=}\PY{l+s+s2}{\PYZdq{}}\PY{l+s+s2}{hist}\PY{l+s+s2}{\PYZdq{}}\PY{p}{,}
    \PY{n+nb}{min}\PY{o}{=}\PY{o}{\PYZhy{}}\PY{l+m+mf}{0.01}\PY{p}{,}
    \PY{n+nb}{max}\PY{o}{=}\PY{l+m+mf}{0.01}\PY{p}{,}
\PY{p}{)}
\PY{n}{hp}\PY{o}{.}\PY{n}{graticule}\PY{p}{(}\PY{p}{)}
\end{Verbatim}
\end{tcolorbox}

    \begin{center}
    \adjustimage{max size={0.9\linewidth}{0.9\paperheight}}{output_97_0.png}
    \end{center}
    { \hspace*{\fill} \\}
    
    \begin{tcolorbox}[breakable, size=fbox, boxrule=1pt, pad at break*=1mm,colback=cellbackground, colframe=cellborder]
\prompt{In}{incolor}{125}{\boxspacing}
\begin{Verbatim}[commandchars=\\\{\}]
\PY{n}{hp}\PY{o}{.}\PY{n}{mollview}\PY{p}{(}
    \PY{n}{map\PYZus{}sample} \PY{o}{\PYZhy{}} \PY{n}{new\PYZus{}map}\PY{p}{,}
    \PY{n}{coord}\PY{o}{=}\PY{p}{[}\PY{l+s+s2}{\PYZdq{}}\PY{l+s+s2}{G}\PY{l+s+s2}{\PYZdq{}}\PY{p}{]}\PY{p}{,}
    \PY{n}{title}\PY{o}{=}\PY{l+s+s2}{\PYZdq{}}\PY{l+s+s2}{Map Difference Galactic Sample 30Hz}\PY{l+s+s2}{\PYZdq{}}\PY{p}{,}
    \PY{n}{unit}\PY{o}{=}\PY{l+s+s2}{\PYZdq{}}\PY{l+s+s2}{mK}\PY{l+s+s2}{\PYZdq{}}\PY{p}{,}
    \PY{n}{norm}\PY{o}{=}\PY{l+s+s2}{\PYZdq{}}\PY{l+s+s2}{hist}\PY{l+s+s2}{\PYZdq{}}\PY{p}{,}
    \PY{c+c1}{\PYZsh{}min=\PYZhy{}0.0,}
    \PY{c+c1}{\PYZsh{}max=0.01,}
\PY{p}{)}
\PY{n}{hp}\PY{o}{.}\PY{n}{graticule}\PY{p}{(}\PY{p}{)}
    

     
        
    
    
      
\end{Verbatim}
\end{tcolorbox}

    \begin{center}
    \adjustimage{max size={0.9\linewidth}{0.9\paperheight}}{output_98_0.png}
    \end{center}
    { \hspace*{\fill} \\}
    
    \hypertarget{devoir-change-lmax-and-rerun}{%
\subsection{Devoir: change lmax and
rerun}\label{devoir-change-lmax-and-rerun}}

    \begin{tcolorbox}[breakable, size=fbox, boxrule=1pt, pad at break*=1mm,colback=cellbackground, colframe=cellborder]
\prompt{In}{incolor}{ }{\boxspacing}
\begin{Verbatim}[commandchars=\\\{\}]

\end{Verbatim}
\end{tcolorbox}

    \begin{tcolorbox}[breakable, size=fbox, boxrule=1pt, pad at break*=1mm,colback=cellbackground, colframe=cellborder]
\prompt{In}{incolor}{ }{\boxspacing}
\begin{Verbatim}[commandchars=\\\{\}]

\end{Verbatim}
\end{tcolorbox}

    \begin{tcolorbox}[breakable, size=fbox, boxrule=1pt, pad at break*=1mm,colback=cellbackground, colframe=cellborder]
\prompt{In}{incolor}{ }{\boxspacing}
\begin{Verbatim}[commandchars=\\\{\}]

\end{Verbatim}
\end{tcolorbox}

    \begin{tcolorbox}[breakable, size=fbox, boxrule=1pt, pad at break*=1mm,colback=cellbackground, colframe=cellborder]
\prompt{In}{incolor}{ }{\boxspacing}
\begin{Verbatim}[commandchars=\\\{\}]

\end{Verbatim}
\end{tcolorbox}

    \begin{tcolorbox}[breakable, size=fbox, boxrule=1pt, pad at break*=1mm,colback=cellbackground, colframe=cellborder]
\prompt{In}{incolor}{ }{\boxspacing}
\begin{Verbatim}[commandchars=\\\{\}]

\end{Verbatim}
\end{tcolorbox}


    % Add a bibliography block to the postdoc
    
    
    
\end{document}
